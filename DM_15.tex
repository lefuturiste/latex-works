\documentclass{article}
\usepackage[english]{babel}
\usepackage[utf8]{inputenc}
\usepackage{fancyhdr}
\usepackage{amsmath}
\usepackage{amsfonts}
\usepackage{mathtools}
\usepackage{indentfirst}
\usepackage{hyperref}
\hypersetup{
    colorlinks=true,
    linkcolor=blue,
    filecolor=magenta,      
    urlcolor=blue,
}
\urlstyle{same}


\usepackage{geometry}
 \geometry{
 a4paper,
 left=20mm,
 top=20mm,
 bottom=20mm,
 right=20mm
 }

\pagestyle{fancy}
\fancyhf{}
\rhead{}
\lhead{DM 15, ex 2 p 37}
\rfoot{Page \thepage}


\makeatletter
\def\@seccntformat#1{%
  \expandafter\ifx\csname c@#1\endcsname\c@section\else
  \csname the#1\endcsname\quad
  \fi}
\makeatother


\begin{document}

\iffalse
Énoncé (exercice 2) : 
\href{https://www.sujetdebac.fr/annales-pdf/2019/s-mathematiques-obligatoire-2019-antilles-guyane-sujet-officiel.pdf}{https://www.sujetdebac.fr/annales-pdf/2019/s-mathematiques-obligatoire-2019-antilles-guyane-sujet-officiel.pdf}
\fi

\section{Partie A}

\textbf{1.}

\vspace{2mm}

\noindent On calcule le vecteur $\overrightarrow{AB}$ :

\vspace{2mm}
$\overrightarrow{AB}
\begin{pmatrix}
   x_B-x_A\\
   y_B-y_A\\
   z_B-z_A
\end{pmatrix}
=
\begin{pmatrix}
   2-2\\
   6-4\\
   \frac{3}{4}-\frac{1}{4}
\end{pmatrix}
= 
\begin{pmatrix}
   0\\
   2\\
   \frac{1}{2}
\end{pmatrix}
$
\vspace{2mm}

\noindent On obtient ainsi une représentation paramétrique de $(AB)$, avec a, b et c les coordonnées d'un point de $(AB)$:

\vspace{2mm}

$(AB):\left\{\begin{matrix}
x = 0 + a\\
y = 2t + b \\
z = \frac{1}{2}t + c
\end{matrix}
\right., \{ a, b, c, t \} \in \mathbb{R}^4
$

\vspace{1mm}

\noindent On injecte les coordonnées du point $A\begin{pmatrix}
   2\\
   4\\
   \frac{1}{4}
\end{pmatrix}$ :
\vspace{1mm}

$(AB):\left\{\begin{matrix*}[l]
x = 2\\
y = 2t + 4 \\
z = \frac{1}{2}t + \frac{1}{4}
\end{matrix*}
\right., t \in \mathbb{R}
$

\vspace{6mm}

\textbf{2.a}

\vspace{2mm}

\noindent D'après la figure, $\overrightarrow{PQ}$ et $\overrightarrow{PU}$ sont 2 vecteurs non colinéaires de $(PQU)$.

\vspace{2mm}

$\overrightarrow{PQ}\begin{pmatrix}
   x_Q-x_P\\
   y_Q-y_P\\
   z_Q-z_P
\end{pmatrix}
=
\begin{pmatrix}
   0\\
   11 - 10\\
   1 - 0
\end{pmatrix}
=
\begin{pmatrix}
   0\\
   1\\
   1
\end{pmatrix}
$

\vspace{2mm}

$\overrightarrow{PQ} \cdot \overrightarrow{n} = 0+1\times1+1\times(-1) = 0 + 1 - 1 = 0 $ \quad Donc $\overrightarrow{PQ} \perp \overrightarrow{n}$

\vspace{2mm}

$\overrightarrow{PU}\begin{pmatrix}
   x_U-x_P\\
   y_U-y_P\\
   z_U-z_P
\end{pmatrix}
=
\begin{pmatrix}
   10 - 0\\
   10 - 10\\
   0
\end{pmatrix}
=
\begin{pmatrix}
   10\\
   0\\
   0
\end{pmatrix}
$

\vspace{2mm}

$\overrightarrow{PU} \cdot \overrightarrow{n} = 10 \times 0 + 0 \times 1 + 0 \times (-1) = 0+0+0 = 0$ \quad Donc $\overrightarrow{PU} \perp \overrightarrow{n}$

\vspace{2mm}

\noindent Donc le vecteur $\overrightarrow{n}$ est un vecteur normal au plan $(PQU)$ car $\overrightarrow{n}$ est orthogonal à deux vecteurs non colinéaires de $(PQU)$. 
\vspace{2mm}

\vspace{6mm}
\textbf{2.b}
\vspace{2mm}

\noindent Une équation cartésienne du plan $(PQU)$ est de la forme :

\vspace{2mm}

$(PQU) : ax+ by + cz + d = 0$

\vspace{2mm}

\noindent  où a, b et c sont les coordonnées d'un vecteur normal au plan, ce qui donne :

\vspace{2mm}

$(PQU): y - z + d = 0$

\vspace{2mm}

\noindent  Pour trouver d, on injecte les coordonnées de $P \in (PQU)$ :

\vspace{2mm}

$10 - 0 + d = 0 \Leftrightarrow 10 + d = 0 \Leftrightarrow d = 10$

\vspace{2mm}

\noindent Donc une équation cartésienne de $(PQU)$ est :

\vspace{2mm}

$(PQU): y - z - 10 = 0$

\vspace{2mm}

\newpage

\textbf{3.}

\vspace{2mm}

\noindent On regarde le point en commun entre $(PQU)$ et $(AB)$ à l'aide de l'équation cartésienne de plan et de la représentation paramétrique.

\vspace{2mm}

$\left\{\begin{matrix*}[l]
   y - z -10 = 0\\
   \\
   x = 2\\
   \\
   y = 2t + 4\\
   \\
   z = \frac{1}{2}t+\frac{1}{4}
\end{matrix*}\right.
\Leftrightarrow
\left\{\begin{matrix*}[l]
   2t + 4 - \frac{1}{2}t - \frac{1}{4} - 10 = 0\\
   \\
   x = 2\\
   \\
   y = 2t + 4\\
   \\
   z = \frac{1}{2}t+\frac{1}{4}
\end{matrix*}\right.
\Leftrightarrow
\left\{\begin{matrix*}[l]
   \frac{3}{2}t - \frac{25}{4} = 0\\
   \\
   x = 2\\
   \\
   y = 2t + 4\\
   \\
   z = \frac{1}{2}t + \frac{1}{4}
\end{matrix*}\right.
$

\vspace{2mm}

$
\Leftrightarrow
\left\{\begin{matrix*}[l]
   t = \displaystyle\frac{\frac{25}{4}}{\frac{3}{2}} = \frac{25}{6}\\
   \\
   x = 2\\
   \\
   y = \displaystyle\frac{25}{3} + 4 = \frac{37}{3}\\
   \\
   z = \displaystyle\frac{25}{12} + \frac{1}{4} = \frac{7}{3}
\end{matrix*}\right.
$

\vspace{2mm}

\noindent Le point en commun entre le plan $(PQU)$ et la droite $(AB)$ a pour coordonnées $\Big(2; \displaystyle\frac{37}{3}; \frac{7}{3}\Big)$. On appellera ce point $I$ par la suite.

\vspace{6mm}

\textbf{4.}

\vspace{2mm}

\noindent Le point $I$ a une côte de 2,33 m ce qui est plus haut que l'obstacle qui a comme côte maximale 1 m. Donc le drone d'Alex ne rencontrera pas d'obstacles.

\section{Partie B}

\textbf{1.}

\vspace{2mm}

$\overrightarrow{AM} = a \cdot \overrightarrow{AB}
\Leftrightarrow \left\{\begin{matrix*}[l]
   x_M - 2 = a \times 0\\
   y_M - 4 = 2a\\
   z_M - \frac{1}{4} = \frac{1}{2}a
\end{matrix*}\right.
\Leftrightarrow \left\{\begin{matrix*}[l]
   x_M = 2\\
   y_M = 2a + 4\\
   z_M = \frac{1}{2} a + \frac{1}{4}
\end{matrix*}\right.$


\vspace{2mm}


$\overrightarrow{CN} = b \cdot \overrightarrow{CD}
\Leftrightarrow \left\{\begin{matrix*}[l]
   x_N - 4 = -2b\\
   y_N - 6 = 0\\
   z_N - \frac{1}{4} = 0
\end{matrix*}\right.
\Leftrightarrow \left\{\begin{matrix*}[l]
   x_N = -2b + 4\\
   y_N = 6\\
   z_N = \frac{1}{4}
\end{matrix*}\right.$

\vspace{2mm}

\noindent On en conclut alors les coordonnées du vecteur $\overrightarrow{MN}$ :

\vspace{3mm}

$\overrightarrow{MN}\begin{pmatrix}
   x_N - x_M\\
   y_N - y_M\\
   z_N - z_M
\end{pmatrix}
= 
\begin{pmatrix}
   -2b + 4 - 2\\
   6 - (2a+4)\\
   \frac{1}{4} - (\frac{1}{2}a + \frac{1}{4})
\end{pmatrix}
=
\begin{pmatrix}
   -2b + 2\\
   2 - 2a\\
   -\frac{1}{2}a
\end{pmatrix}
$

\vspace{6mm}

\textbf{2.}

\vspace{2mm}

\noindent On cherche alors les valeurs de a et b pour lesquels la droite $(MN)$ est perpendiculaire à la droite $(AB)$ et à la droite $(CD)$.

\vspace{2mm}

$\overrightarrow{MN} \cdot \overrightarrow{AB} = 0$

\vspace{2mm}

$\Leftrightarrow (-2b+2)\times 0 + 2(2-2a) + \displaystyle\frac{1}{2} \times -\frac{1}{2} a = 0$

\vspace{2mm}

$\Leftrightarrow 4 - 4a - \displaystyle\frac{1}{4} a = 0$

\vspace{2mm}

$\Leftrightarrow 4 - \displaystyle\frac{17}{4}a = 0$

\vspace{2mm}

$\Leftrightarrow a = \displaystyle\frac{16}{17}$

\vspace{4mm}

$\overrightarrow{MN} \cdot \overrightarrow{CD} = 0$

\vspace{2mm}

$\Leftrightarrow -2(-2b + 2) = 0$

\vspace{2mm}

$\Leftrightarrow 4b - 4 = 0$

\vspace{2mm}

$\Leftrightarrow 4b = 4$

\vspace{2mm}

$\Leftrightarrow b = 1$

\vspace{2mm}

Donc la distance $MN$ est minimale quand $a = \frac{16}{17}$ et $b = 1$

\vspace{6mm}

\textbf{3.}

\vspace{2mm}

\noindent Avec $a = \frac{16}{17}$ et $b = 1$, $\overrightarrow{MN}$ a pour coordonnées :

\vspace{4mm}

$\overrightarrow{MN}\begin{pmatrix}
   -2 \times 1 + 2\\
   \\
   2 - 2 \times \displaystyle\frac{16}{17}\\
   \\
   -\frac{1}{2}\times\displaystyle\frac{16}{17}
\end{pmatrix}
= 
\begin{pmatrix}
   0\\
   \\
   \displaystyle\frac{2}{17}\\
   \\
   \displaystyle\frac{-8}{17}
\end{pmatrix}
$

\vspace{4mm}

\noindent On calcule ensuite la norme de $\overrightarrow{MN}$ :

\vspace{2mm}

$||\overrightarrow{MN}|| = MN = \sqrt{0 + \Big(\displaystyle\frac{2}{17}\Big)^2 + \Big(-\frac{8}{17}\Big)^2} = \sqrt{\displaystyle\frac{4}{289} + \frac{64}{289}} = \sqrt{\displaystyle\frac{4}{17}} = \displaystyle\frac{2}{\sqrt{17}}$

\vspace{2mm}

$MN \simeq 0,49$

\vspace{2mm}

\noindent Il est précisé qu'une unité du repère du modèle correspond à 10 m dans la réalité. Si on multiplie par 10 la distance $MN$ on obtient 4,9 m ce qui est bien supérieur à la distance de sécurité. La consigne est donc respectée.

\end{document}


\documentclass{article}
\usepackage[english]{babel}
\usepackage[utf8]{inputenc}
\usepackage{fancyhdr}
\usepackage{amsmath}
\usepackage{amsfonts}
\usepackage{mathrsfs}
\usepackage{mathtools}
\usepackage{indentfirst}
\usepackage{hyperref}
\usepackage{tikz,amsmath}
\usetikzlibrary{trees}

\hypersetup{
    colorlinks=true,
    linkcolor=blue,
    filecolor=magenta,
    urlcolor=blue,
}
\urlstyle{same}

\parskip 1ex

\usepackage{geometry}
 \geometry{
 a4paper,
 left=20mm,
 top=20mm,
 bottom=20mm,
 right=20mm
 }

\pagestyle{fancy}
\fancyhf{}
\rhead{DM 21}%(corrigé)
\lhead{Ex 1 p.27}
\rfoot{Page \thepage}

\makeatletter
\def\@seccntformat#1{%
  \expandafter\ifx\csname c@#1\endcsname\c@section\else
  \csname the#1\endcsname\quad
  \fi}
\makeatother

\newcommand{\vspacem}{\vspace{2mm}}
\newcommand{\bfrac}[2]{\displaystyle\frac{#1}{#2}}

\newif\ifquoteopen
\catcode`\"=\active % lets you define `"` as a macro
\DeclareRobustCommand*{"}{%
   \ifquoteopen
     \quoteopenfalse ''%
   \else
     \quoteopentrue ``%
   \fi
}

\begin{document}

%https://www.sujetdebac.fr/annales-pdf/2019/s-mathematiques-obligatoire-2019-amerique-nord-corrige.pdf

\subsection*{Partie A}

\textbf{1.a}
\vspacem

\noindent Vu que la variable aléatoire $X$ suit la loi normale avec les paramètres $\mu = 1.5$ et $\sigma = 0.07$ on calcule la probabilité que X soit compris dans l'intervalle de validité : $P(1.35 \leq X \leq 1.65)$

\noindent Grâce à la commande $normalFRep(1.35, 1.65, 1.5, 0.07)$  sur la calculatrice on obtient :

$P(1.35 \leq X \leq 1.65) = 0.968$

\noindent La probabilité que le tube soit accepté au contrôle est donc de $0.968$.

\vspacem
\textbf{1.b}

\noindent $X_1$ suit la loi normale $N(1.5n \sigma_1)$

\noindent On connaît $P(1.35 \leq x \leq 1.65) = 0.98$

\noindent On remarque le fait que l'espérance $1.5$ est au milieu de l'intervalle de validité de l'épaisseur d'un tube $[1.35, 1.65]$ car $\bfrac{1.35+1.65}{2} = 1.5$. On peut donc établir la relation : 

$P(1.35 \leq X \leq 1.65) = 2 \times P(1.5 \leq X \leq 1.65)$

$\Leftrightarrow 0.98 = 2 \times (P(X \leq 1.65) - P(X \leq 1.5))$

$\Leftrightarrow 0.98 = 2(P(X\leq 1.65) - 0.5)$

$\Leftrightarrow 0.98 = 2P(X \leq 1.65) - 2\times 0.5$

$\Leftrightarrow 0.98 = 2P(X \leq 1.65) - 1$

$\Leftrightarrow \bfrac{1.98}{2} = P(X \leq 1.65)$

$\Leftrightarrow 0.99 = P(X \leq 1.65)$

\noindent On transforme la probabilité $P(X \leq 1.65)$ dans un contexte de loi normale centré réduite:

$P\Big(Z \leq \bfrac{0.15}{\sigma_1}\Big) = 0.99$

\noindent On connaît donc l'aire sous la courbe sur l'intervalle $\Big]-\infty, \bfrac{0.15}{\sigma_1}\Big]$, on utilise alors une fonction de la calculatrice pour trouver la valeur de la borne supérieure :

$FracNormale(0.99, 0, 1) = 2.326$

\noindent On en déduit alors la valeur de $\sigma_1$

$\bfrac{0.15}{\sigma_1} = 2.326 \Leftrightarrow \bfrac{0.15}{\sigma_1} = 2.326$

$\Leftrightarrow \sigma_1 = \bfrac{1}{2.326} \times 0.15 \Leftrightarrow \sigma_1 = 0.064$

\vspacem
\textbf{2.a}

\noindent On a la probabilité $p = 0.02$ (ou proportion) pour l'évènement $"Avoir un tube non conforme pour la longueur"$ donc $q = 1-p = 1-0.02 = 0.98$ (la probabilité de l'événement contraire).

\noindent On a $n = 250$ ainsi que $F_{250} = \bfrac{10}{250} = 0.04$

\noindent On utilise la définition de l'intervalle de fluctuation asymptotique de $F_n$ au seuil $95\%$ :

$I = \Big[p - 1.96\bfrac{\sqrt{pq}}{n}; p + 1.96\bfrac{\sqrt{pq}}{\sqrt{n}} \Big]$

\noindent On calcule le terme commun aux deux borne de manière indépendante, on appelle ce facteur $h$:

$h = 1.96\bfrac{\sqrt{pq}}{n} = 1.96\bfrac{\sqrt{0.02 \times 0.98}}{\sqrt{250}} = 0.017$

$I = [0.02 - 0.017; 0.02 + 0.017] = [0.003; 0.037]$

\vspacem
\textbf{2.b}

\noindent On remarque que la fréquence $F_{250} = 0.04$ obtenue n'est pas contenue dans l'intervalle de fluctuation calculé à la question $2.a$. Dans ce cas il faut réviser la machine.

\subsection*{Partie B}

\textbf{1.}

\tikzstyle{level 1}=[level distance=3.5cm, sibling distance=3.5cm]
\tikzstyle{level 2}=[level distance=3.5cm, sibling distance=2cm]
\tikzstyle{bag} = [text width=4em, text centered]
\tikzstyle{end} = [circle, minimum width=3pt,fill, inner sep=0pt]
\begin{tikzpicture}[grow=right]
\node[bag] {\space}
    child {
        node[bag] {$\bar{E}$}
            child {
                node[end, label=right:
                    {$\bar{L}$}] {}
                edge from parent
                node[midway,fill=white]  {$0.1$}
            }
            child {
                node[end, label=right:
                    {$L$}] {}
                edge from parent
                node[midway,fill=white]  {$0.9$}
            }
            edge from parent 
            node[midway,fill=white]  {$0.04$}
    }
    child {
        node[bag] {$E$}
            child {
                node[end, label=right:
                    {$\bar{L}$}] {}
                edge from parent
                node[midway,fill=white]  {$0.05$}
            }
            child {
                node[end, label=right:
                    {$L$}] {}
                edge from parent
                node[midway,fill=white]  {$0.95$}
            }
        edge from parent         
            node[midway,fill=white]  {$0.96$}
    };
\end{tikzpicture}

\noindent D'après l'énoncé : $P(L \cap \bar{E}) = 0.036$

\noindent On cherche $P_{\bar{E}}(L) = \bfrac{P(L \cap \bar{E})}{P(\bar{E})} = \bfrac{0.036}{0.04} = 0.9$

\noindent Donc $P_{\bar{E}}(\bar{L}) = 1 - P_{\bar{E}}(L) = 1 - 0.9 = 0.1$

\vspacem
\textbf{2.}

\noindent On utilise la formule des probabilités totales :

$P(L) = P_E(L) \times P(E) + P_{\bar{E}}(L) \times P(\bar{E}) = 0.96 \times 0.95 + 0.04 \times 0.9 = 0.948$


\vspace{2mm}
\end{document}

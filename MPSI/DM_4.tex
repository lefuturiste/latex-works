\documentclass{article}
\usepackage[english]{babel}
\usepackage[utf8]{inputenc}
\usepackage{fancyhdr}
\usepackage{amsmath}
\usepackage{amsfonts}
\usepackage{mathrsfs}
\usepackage{mathtools}
\usepackage{indentfirst}
\usepackage{hyperref}
\usepackage{tikz,amsmath}
\usepackage{tikz,tkz-tab,amsmath}
\usetikzlibrary{trees}
\usetikzlibrary{arrows}
\usepackage{enumerate}

%\setlength{\TPHorizModule}{1mm}
%\setlength{\TPVertModule}{1mm}
  
\hypersetup{
    colorlinks=true,
    linkcolor=blue,
    filecolor=magenta,
    urlcolor=blue,
}
\urlstyle{same}

\parskip 1ex

\usepackage{geometry}
 \geometry{
 a4paper,
 left=20mm,
 top=20mm,
 bottom=20mm,
 right=20mm
 }
\usepackage{multicol}
\pagestyle{fancy}
\fancyhf{}
\lhead{Matthieu Bessat}
\rhead{DM 4}
\rfoot{Page \thepage}

\makeatletter
\def\@seccntformat#1{%
  \expandafter\ifx\csname c@#1\endcsname\c@section\else
  \csname the#1\endcsname\quad
  \fi}
\makeatother

\newcommand{\vspacem}{\vspace{2mm}}
\newcommand{\bfrac}[2]{\displaystyle\frac{#1}{#2}}
\newcommand{\bbinom}[2]{\displaystyle\binom{#1}{#2}}

\newif\ifquoteopen
\catcode`\"=\active % lets you define `"` as a macro
\DeclareRobustCommand*{"}{%
   \ifquoteopen
     \quoteopenfalse ''%
   \else
     \quoteopentrue ``%
   \fi
}

\begin{document}

\subsection*{Étude d'une équation d'ordre 2}

\textbf{1.a}

\noindent La forme normalisé de l'équation $(\mathcal{E})$ est :

$\mathcal{(E)}: y'+\bfrac{1}{t}y=-\bfrac{\sin(t)}{t}$

\noindent Sous cette forme on constate que les intervalles d'études sont $\mathbb{R}^*_-$ et $\mathbb{R}^*_+$.

\textbf{1.b}

\noindent L'équation homogène de $(\mathcal{E})$ est $(\mathcal{E}_h): ty'+y=0$

\noindent Les solutions à l'équation homogènes sont :

$S_{(\mathcal{E}_h)} = \{ t \mapsto Ce^{-A(t)}, c \in \mathbb{R} \}$

\noindent Avec $A(t)$ une primitive de la fonction $\bfrac{b(t}{a(t)} = \bfrac{1}{t}$

\noindent Ainsi $A(t) = \ln(t)$, donc :

$S_{(\mathcal{E}_h)} = \{ t \mapsto Ce^{-ln(t)}, C \in \mathbb{R} \}$

$S_{(\mathcal{E}_h)} = \{ t \mapsto C\bfrac{1}{t}, C \in \mathbb{R} \}$

\textbf{1.c}

\noindent Recherche d'une solution particulière en utilisant la méthode de variation de la constante.

\noindent Sous la forme :  $y: t \mapsto k(t)\bfrac{1}{t}$

$y': t \mapsto k'(t)\bfrac{1}{t} + \Big(\bfrac{1}{t}\Big)'k(t)$

$y': t \mapsto \bfrac{k'(t)}{t} - \bfrac{k(t)}{t^2}$

$y': t \mapsto \bfrac{tk'(t)-k(t)}{t^2}$

\noindent En réinjectant dans la forme normalisé de $(\mathcal{E})$, on obtient :

$ty'+y = -\sin(t)$

$\Leftrightarrow t\Big(\bfrac{tk'(t)-k(t)}{t^2}\Big) + k(t)\bfrac{1}{t} = -\sin(t)$

$\Leftrightarrow \bfrac{tk'(t) - k(t)}{t} + k(t)\bfrac{1}{t} = -\sin(t)$

$\Leftrightarrow \bfrac{tk'(t)}{t} + k(t)\bfrac{1}{t} = -\sin(t)$

$\Leftrightarrow k'(t) = -sin't)$

\noindent Ainsi $k(t) = \cos(t)$

\noindent Donc une solution particulière est : $y \mapsto \cos(t) \bfrac{1}{t}$

\textbf{1.c}

\noindent Solutions générales = Solution particulière + Solutions homogènes

\noindent Donc la forme générale des solutions de $(\mathcal{E})$

$S_{(\mathcal{E})} = \big\{ t \mapsto \cos(t)\bfrac{1}{t} + \alpha\bfrac{1}{t}, \alpha \in \mathbb{R} \big\}$

$S_{(\mathcal{E})} = \big\{ t \mapsto \bfrac{\cos(t) + \alpha}{t}, \alpha \in \mathbb{R} \big\}$

\textbf{1.e}

\noindent On a ainsi résolu $(\mathcal{E})$ sur $\mathbb{R}^*_-$ et $\mathbb{R}^*_+$ réunies c'est à dire sur $\mathbb{R}^*$

% sqcup

Soit $y_1$ et $y_2$ deux fonctions solutions de $(\mathcal{E})$ respectivement sur les intervalles $\mathbb{R}^*_-$ et $\mathbb{R}^*_+$

Sur $\mathbb{R}^*_-$, on a :

$y_1(t) = \bfrac{A_1 + \cos(t)}{t}$

$y_1(t) = \bfrac{A_1 + 2\sin(\frac{t}{2})^2 + 1}{t}$

$y_1(t) = \bfrac{A_1 + 1}{t} + \bfrac{2 \sin(\frac{t}{2})^2}{t}$

Pour que $y_1$ admette une limite finie en $t = 0$, il faut que $A_1 = -1$

\end{document}


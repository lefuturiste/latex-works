\documentclass{article}
\usepackage[english]{babel}
\usepackage[utf8]{inputenc}
\usepackage{fancyhdr}
\usepackage{amsmath}
\usepackage{amsfonts}
\usepackage{mathrsfs}
\usepackage{mathtools}
\usepackage{indentfirst}
\usepackage{hyperref}
\usepackage{tikz,amsmath}
\usepackage{tikz,tkz-tab,amsmath}
\usetikzlibrary{trees}
\usetikzlibrary{arrows}
\usepackage{cancel}
\usepackage{xfrac}  

\usepackage{enumerate}

%\setlength{\TPHorizModule}{1mm}
%\setlength{\TPVertModule}{1mm}
  
\hypersetup{
    colorlinks=true,
    linkcolor=blue,
    filecolor=magenta,
    urlcolor=blue,
}
\urlstyle{same}

\parskip 1ex

\usepackage{geometry}
 \geometry{
 a4paper,
 left=20mm,
 top=20mm,
 bottom=20mm,
 right=20mm
 }
\usepackage{multicol}
\pagestyle{fancy}
\fancyhf{}
\lhead{Matthieu Bessat}
\rhead{DM 6 - 23 nov. 2020}
\rfoot{Page \thepage}

\makeatletter
\def\@seccntformat#1{%
  \expandafter\ifx\csname c@#1\endcsname\c@section\else
  \csname the#1\endcsname\quad
  \fi}
\makeatother

\makeatletter
\renewcommand*\env@matrix[1][*\c@MaxMatrixCols c]{%
  \hskip -\arraycolsep
  \let\@ifnextchar\new@ifnextchar
  \array{#1}}
\makeatother

\newcommand{\vspacem}{\vspace{3mm}}
\newcommand{\bfrac}[2]{\displaystyle\frac{#1}{#2}}
\newcommand{\bbinom}[2]{\displaystyle\binom{#1}{#2}}

\newcommand\aug{\fboxsep=-\fboxrule\!\!\!\fbox{\strut}\!\!\!}

\newif\ifquoteopen
\catcode`\"=\active % lets you define `"` as a macro
\DeclareRobustCommand*{"}{%
   \ifquoteopen
     \quoteopenfalse ''%
   \else
     \quoteopentrue ``%
   \fi
}

\begin{document}

\section*{DM 6}

\textbf{1.a}

\setlength{\abovedisplayskip}{0pt}
\setlength{\belowdisplayskip}{0pt}
\setlength{\abovedisplayshortskip}{0pt}
\setlength{\belowdisplayshortskip}{0pt}
\begin{flalign*}
A^2 &= 
\begin{pmatrix}
2 & 1  & -3 \\
1 & -2 & -1 \\
3 & 1  & -4
\end{pmatrix}
\begin{pmatrix}
2 & 1  & -3 \\
1 & -2 & -1 \\
3 & 1  & -4
\end{pmatrix} =
\begin{pmatrix} 
-4 & -3 & 5 \\
-3 &  4 & 3 \\
-5 & -3 & 6
\end{pmatrix}
&&\\
&
\end{flalign*}

\vspace{-6px}

\noindent Détail du coefficient $a^2_{1,1}$ : $a^2_{1,1} = 2\times2 + 1\times1 + (-3)\times3 = 4+1-9 = -4$

\noindent On a donc : 

\vspace{-10px}

\[
\boxed{
    A^2 = 
    \begin{pmatrix} 
    -4 & -3 & 5 \\
    -3 &  4 & 3 \\
    -5 & -3 & 6
    \end{pmatrix}
}
\]

\vspacem
\vspacem
\textbf{1.b}

\noindent On utilise la méthode du formalisme matriciel avec le pivot de Gauss :

\begin{flalign*}
&\begin{pmatrix}[ccc|ccc]
\boxed{1} & -1 & -1 & 1 & 0 & 0 \\
0 & 1 & 1 & 0 & 1 & 0 \\
1 & -2 & -1 & 0 & 0 & 1 \\
\end{pmatrix}
\quad
\begin{matrix*}[l]
L_3 \leftarrow L_3 - L_1
\end{matrix*}
&&\\
& \text{On trouve 4 pivots donc $P$ est inversible.}
&&\\
&\begin{pmatrix}[ccc|ccc]
1 & -1  & -1      & 1  & 0 & 0 \\
0 & \boxed{1} & 1 & 0  & 1 & 0 \\
0 & -1 & 0        & -1 & 0 & 1 \\
\end{pmatrix}
\quad
\begin{matrix*}[l]
L_1 \leftarrow L_1 + L_2 \\
L_3 \leftarrow L_3 + L_2
\end{matrix*}
&&\\
&\begin{pmatrix}[ccc|ccc]
1 & 0 & 0         & 1 & 1 & 0  \\
0 & 1 & 1         & 0 & 1 & 0  \\
0 & 0 & \boxed{1} & -1 & 1 & 1 \\
\end{pmatrix}
\quad
\begin{matrix*}[l]
L_2 \leftarrow L_2 - L_3
\end{matrix*}
&&\\
&\begin{pmatrix}[ccc|ccc]
1 & 0 & 0 & 1 & 1 & 0  \\
0 & 1 & 0 & 1 & 0 & -1  \\
0 & 0 & 1 & -1 & 1 & 1 \\
\end{pmatrix}
\end{flalign*}

\vspace{5mm}

\noindent Ainsi :

\vspace{-15mm}

\[
\boxed{
    P^{-1}=\begin{pmatrix}
    1 & 1 & 0  \\
    1 & 0 & -1  \\
    -1 & 1 & 1 \\
    \end{pmatrix}
}
\]

\vspacem
\vspacem
\vspacem

\textbf{1.c}
\vspace{3mm}
\begin{flalign*}
T &= P^{-1}AP = \begin{pmatrix}
1 & 1 & 0  \\
1 & 0 & -1  \\
-1 & 1 & 1 \\
\end{pmatrix}\times\begin{pmatrix}
2 & 1 & -3  \\
1 & -2 & -1  \\
3 & 1 & -4 \\
\end{pmatrix}\times P
&&\\
T &= \begin{pmatrix}
3  & -1 & -4  \\
-1 &  0 & 1  \\
2  & -2 & -2 \\
\end{pmatrix}\times P
&&\\
T &= \begin{pmatrix}
3  & -1 & -4  \\
-1 &  0 & 1  \\
2  & -2 & -2 \\
\end{pmatrix}\times \begin{pmatrix}
1  & -1 & -1  \\
0 &  1 & 1  \\
1  & -2 & -1 \\
\end{pmatrix}
&&\\
T &= \begin{pmatrix}
-1 & 4 & 0  \\
0 & -1 & 0  \\
0  & 0 & -2 \\
\end{pmatrix}
\end{flalign*}

\vspace{5mm}

\noindent En conséquence :

\vspace{-15mm}

\[
\boxed{
T = \begin{pmatrix}
-1 & 4 & 0  \\
0 & -1 & 0  \\
0  & 0 & -2 \\
\end{pmatrix}
}
\]

\newpage
\textbf{1.d}
\vspace{5mm}
\begin{flalign*}
T^2 &= T \times T = \begin{pmatrix}
-1 & 4 & 0  \\
0 & -1 & 0  \\
0  & 0 & -2 \\
\end{pmatrix}\times\begin{pmatrix}
-1 & 4 & 0  \\
0 & -1 & 0  \\
0  & 0 & -2 \\
\end{pmatrix} = \begin{pmatrix}
1 & -8 & 0  \\
0 &  1 & 0  \\
0 & 0 & 4 \\
\end{pmatrix}
&&\\
T^3 &= T^2 \times T = \begin{pmatrix}
1 & -8 & 0  \\
0 &  1 & 0  \\
0 & 0 & 4 \\
\end{pmatrix} \times \begin{pmatrix}
-1 & 4 & 0  \\
0 & -1 & 0  \\
0  & 0 & -2 \\
\end{pmatrix} = \begin{pmatrix}
-1 & 12 & 0  \\
0 & -1 & 0  \\
0  & 0 & -8 \\
\end{pmatrix}
\end{flalign*}

\vspacem
\textbf{1.e}

\noindent On conjecture l'expression de $T^n$ :

\begin{equation*}
\forall n \in \mathbb{N}, \quad
T^n = (-1)^n 
\begin{pmatrix}
1 & -4n & 0 \\
0 & 1   & 0 \\
0 & 0   & 2^n
\end{pmatrix}
\end{equation*}

\noindent On utilise le raisonnement par récurrence :

\noindent \textbf{\underline{Initialisation :}} pour $n = 0$

$T^0 = (-1)^0\begin{pmatrix}
1 & -4\times0 & 0 \\
0 & 1   & 0 \\
0 & 0   & 2^0
\end{pmatrix} = \begin{pmatrix}
1 & 0 & 0 \\
0 & 1 & 0 \\
0 & 0 & 1
\end{pmatrix}$

\noindent Donc la propriété est vrai au rang $n = 0$

\noindent \textbf{\underline{Hérédité :}}
On suppose la propriété vrai à un certain rang $n$. Démontrons la propriété vrai au rang $n+1$ :

\begin{flalign*}
T^n &= (-1)^n\begin{pmatrix}
1 & -4n & 0 \\
0 & 1   & 0 \\
0 & 0   & 2^n
\end{pmatrix}
&&\\
T^n \times T &= \Bigg((-1)^n\begin{pmatrix}
1 & -4n & 0 \\
0 & 1   & 0 \\
0 & 0   & 2^n
\end{pmatrix}\Bigg) \times T
&&\\
T^n \times T &= (-1)^n\begin{pmatrix}
1 & -4n & 0 \\
0 & 1   & 0 \\
0 & 0   & 2^n
\end{pmatrix} \times \begin{pmatrix}
-1 & 4 & 0  \\
0 & -1 & 0  \\
0  & 0 & -2 \\
\end{pmatrix}
&&\\
T^n \times T &= (-1)^n \begin{pmatrix}
-1 & 4+4n & 0  \\
0 & -1 & 0  \\
0 & 0 & -2^{n+1} \\
\end{pmatrix}
&&\\
T^n \times T &= (-1)(-1)^n \begin{pmatrix}
1 & -4(n+1) & 0  \\
0 & -1 & 0  \\
0 & 0 & 2^{n+1} \\
\end{pmatrix}
&&\\
T^{n+1} \times T &= (-1)^{n+1} \begin{pmatrix}
1 & -4(n+1) & 0  \\
0 & -1 & 0  \\
0 & 0 & 2^{n+1} \\
\end{pmatrix}
\end{flalign*}
\vspace{3mm}

\noindent On retrouve bien la forme avec $n+1$ au lieu de $n$.

\noindent Donc la propriété est vrai pour le rang $n+1$.

\vspace{5mm}

\noindent \textbf{\underline{Conclusion :}}

\vspace{-12mm}

\begin{equation*}
\boxed{
\forall n \in \mathbb{N}, \quad
T^n = (-1)^n 
\begin{pmatrix}
1 & -4n & 0 \\
0 & 1   & 0 \\
0 & 0   & 2^n
\end{pmatrix}
}
\end{equation*}

\newpage

\textbf{1.f}

\noindent D'après la question 1.c on a :

\vspace{-4mm}

\begin{flalign*}
\quad & T = P^{-1}AP &&\\
\text{Ainsi :} \quad & T^n = (P^{-1}AP)^n  &&\\
               \quad & T^n = (P^{-1}A\cancel{P})\times(\cancel{P^{-1}}A\cancel{P})\times \cdots \times(\cancel{P^{-1}}A\cancel{P})\times(\cancel{P^{-1}}AP)  &&\\
               \quad & T^n = P^{-1}A^nP  &&\\
\text{D'où :} \quad & A^n = PT^nP^{-1}  &&\\
\quad & A^n =
\begin{pmatrix}
1 & -1 & -1  \\
0 & 1 & 1  \\
1 & -2 & -1 \\
\end{pmatrix}\times
(-1)^n 
\begin{pmatrix}
1 & -4n & 0 \\
0 & 1   & 0 \\
0 & 0   & 2^n
\end{pmatrix}
\times P^{-1}
&&\\
\quad & A^n =
\begin{pmatrix}
(-1)^n & -4(-1)^nn-(-1)^n & -(-1)^n\cdot2^n  \\
0 & (-1)^n & (-1)^n\cdot2^n  \\
(-1)^n & -4(-1)^nn-2(-1)^n & -(-1)^n\cdot2^n \\
\end{pmatrix}\times
\begin{pmatrix}
1 & 1 & 0 \\
1 & 0 & -1  \\
-1 & 1 & 1 \\
\end{pmatrix}
&&\\
\quad & A^n =
\begin{pmatrix}
-4(-1)^nn+(-1)^n\cdot2^n & (-1)^n-(-1)^n\cdot2^n & 4(-1)^nn+(-1)^n-(-1)^n\cdot2^n \\   
(-1)^n-(-1)^n\cdot2^n & (-1)^n\cdot 2^n & (-1)^n\cdot2^n-(-1)^n \\
-4(-1)^nn-(-1)^n+(-1)^n\cdot2^n & (-1)^n-(-1)^n\cdot2^n & 4(-1)^nn + 2(-1)^n - (-1)^n\cdot2^n \\
\end{pmatrix}
&&\\
\end{flalign*}

%
%\begin{pmatrix}
%1 & 1 & 0  \\
%1 & 0 & -1  \\
%-1 & 1 & 1 \\
%\end{pmatrix}

\vspace{-2mm}

\vspace{5mm}

\noindent On a donc : \quad

\vspace{-14mm}

\begin{equation*}
\quad
\boxed{
\forall n \in \mathbb{N}, \quad
A^n = 
\begin{pmatrix}
4n+(-2)^n & (-1)^n - (-2)^n & -4n+(-1)^n-(-2)^n \\
(-1)^n-(-2)^n & (-2)^n   & (-1)^n+(-2)^n \\
-(-1)^n+4n+(-2)^n & (-1)^n-(-2)^n   & -4n+2(-1)^n-(-2)^n
\end{pmatrix}
}
\end{equation*}

\textbf{1.g}

\noindent Pour savoir si A est inversible, on utilise la méthode du formalisme matriciel avec le pivot de Gauss :

\begin{flalign*}
&\begin{pmatrix}[ccc|ccc]
\boxed{2} & 1 & -3 & 1 & 0 & 0 \\
1 & -2 & -1 & 0 & 1 & 0 \\
3 & 1 & -4 & 0 & 0 & 1 \\
\end{pmatrix}
\quad
\begin{matrix*}[l]
\end{matrix*}
&&\\
& \text{On trouve 3 pivots et aucune équations de compatibilités donc $A$ est inversible.}
&&\\
\end{flalign*}

\noindent On suit le même algorithme qu'a la question 1.b et on obtient $A^{-1}$ :

\[
\boxed{
    A^{-1} = 
    \begin{pmatrix} 
    -\sfrac{9}{2} & -\sfrac{1}{2} & \sfrac{7}{2} \\
    -\sfrac{1}{2} & -\sfrac{1}{2} & \sfrac{1}{2} \\
    -\sfrac{7}{2} & -\sfrac{1}{2} & \sfrac{5}{2}
    \end{pmatrix}
}
\]

\vspacem
\textbf{2.a}

\noindent On note $a$, $b$, $c$ les éléments du couple représentant l'image de l'application $f$, de la même manière note $e$, $f$ et $g$, les éléments du couple représentant l'image de l'application $g$

\begin{flalign*}
&\left\{\begin{matrix*}[l]
    d: 2a + b - 3c\\
    e: a - 2b - c \\
    f: 3a + b- 4c \\
\end{matrix*}\right.
&&\\
\Leftrightarrow &\left\{\begin{matrix*}[l]
    d: 2(2x+y-3z) + (x-2y-z) - 3(3x+y-4z)\\
    e: (2x+y-3z)-2(x-2y-z)-(3x+y-4z) \\
    f: 3(2x+y-3z) + (x-2y-z)- 4(3x+y-4z) \\
\end{matrix*}\right.
&&\\
\Leftrightarrow &\left\{\begin{matrix*}[l]
    d: 4x+2y-6z+x-2y-z-9x-3y+12z \\
    e: 2x+y-3z-2x+4y+2z-3x-y+4z \\
    f: 6x+3y-9z+x-2y-z-12x-4y+16z \\
\end{matrix*}\right.
\end{flalign*}

\vspace{5px}

\noindent Ainsi, on trouve l'expression de $g$:

\[
\boxed{
    g : \medspace\left\{\begin{flalign*}
        \mathbb{R}^3 &\longrightarrow \mathbb{R}^3&&\\
        (x, y, z) &\longmapsto (-4x-3y+5y, -3x+4y+3z, -5x-3y+6z)
    \end{flalign*}\right.
}
\]

\textbf{2.b}

\noindent On trouve :

\vspace{-5mm}

\begin{flalign*}
\forall (x,y,z),(d,e,f) \in \mathbb{R}^3, \quad\;  g(x, y, z) = (d, e, f) \quad &\Leftrightarrow \quad A\begin{pmatrix}a \\ b \\ c\end{pmatrix} = \begin{pmatrix}d \\ e \\ f\end{pmatrix}
&&\\
&\Leftrightarrow \quad A\times A\begin{pmatrix}x \\ y \\ z\end{pmatrix} = \begin{pmatrix}d \\ e \\ f\end{pmatrix}
&&\\
&\Leftrightarrow \quad A^2\begin{pmatrix}x \\ y \\ z\end{pmatrix} = \begin{pmatrix}d \\ e \\ f\end{pmatrix}
&&\\
\end{flalign*}

\vspace{-5mm}

\noindent La question 1.a nous donne : $A^2 = 
\begin{pmatrix} 
-4 & -3 & 5 \\
-3 &  4 & 3 \\
-5 & -3 & 6
\end{pmatrix}$

\noindent D'où :

\begin{flalign*}
\begin{pmatrix} 
d \\ e \\ f
\end{pmatrix} =
A^2\times\begin{pmatrix} 
x \\ y \\ z
\end{pmatrix} &= 
\begin{pmatrix} 
-4 & -3 & 5 \\
-3 &  4 & 3 \\
-5 & -3 & 6
\end{pmatrix}
\begin{pmatrix} 
x \\ y \\ z
\end{pmatrix} = \begin{pmatrix} 
-4x -3y + 5y \\
-3x +  4y + 3z \\
-5x -3y + 6z
\end{pmatrix}
&&\\
\end{flalign*}

\textbf{2.c}

\noindent $f$ est bijective car A est inversible. (d'après la question 1.g)

\begin{flalign*}
A\begin{pmatrix}x \\ y \\ z\end{pmatrix} &= \begin{pmatrix}a \\ b \\ c\end{pmatrix}
\quad\Leftrightarrow\quad \cancel{A^{-1}A}\begin{pmatrix}x \\ y \\ z\end{pmatrix} = A^{-1}\begin{pmatrix}a \\ b \\ c\end{pmatrix}
\quad\Leftrightarrow\quad \begin{pmatrix}x \\ y \\ z\end{pmatrix} = A^{-1}\begin{pmatrix}a \\ b \\ c\end{pmatrix}
\end{flalign*}

\vspace{3px}

\noindent \begin{flalign*}
\text{Ainsi :} \quad \forall (a,b,c),(x,y,z) \in \mathbb{R}^3, \quad\;  f^{-1}(a, b, c) = (x, y, z) \quad &\Leftrightarrow \quad A^{-1}\begin{pmatrix}a \\ b \\ c\end{pmatrix} = \begin{pmatrix}x \\ y \\ c\end{pmatrix}
&&\\
\end{flalign*}

\vspace{-20px}

\noindent Sachant, $A^{-1}$ :
$A^{-1} = 
\begin{pmatrix} 
-\sfrac{9}{2} & -\sfrac{1}{2} & \sfrac{7}{2} \\
-\sfrac{1}{2} & -\sfrac{1}{2} & \sfrac{1}{2} \\
-\sfrac{7}{2} & -\sfrac{1}{2} & \sfrac{5}{2}
\end{pmatrix}$

\vspace{10mm}

\noindent On a donc :

\vspace{-15mm}

\[
\boxed{
    f^{-1} : \medspace\left\{\begin{flalign*}
        \mathbb{R}^3 &\longrightarrow \mathbb{R}^3&&\\
        (a, b, c) &\longmapsto \Big(\sfrac{1}{2}\big[-9a-b+7c\big], \sfrac{1}{2}\big[-a-b+c\big], \sfrac{1}{2}\big[-7a-b+5c\big]\Big)
    \end{flalign*}\right.
}
\]

\vspace{5mm}

\textbf{2.d}

\vspace{-5mm}
\begin{flalign*}
f^n(0,1,0) = (u, v, t) \quad &\Leftrightarrow \quad A^n\times\begin{pmatrix}0 \\ 1 \\ 0\end{pmatrix} = \begin{pmatrix}u \\ v \\ t\end{pmatrix}
&&\\ 
&\Leftrightarrow \quad \begin{pmatrix}u \\ v \\ t\end{pmatrix} = \begin{pmatrix}
4n+(-2)^n & (-1)^n - (-2)^n & -4n+(-1)^n-(-2)^n \\
(-1)^n-(-2)^n & (-2)^n   & (-1)^n+(-2)^n \\
-(-1)^n+4n+(-2)^n & (-1)^n-(-2)^n   & -4n+2(-1)^n-(-2)^n
\end{pmatrix}\times\begin{pmatrix}0 \\ 1 \\ 0\end{pmatrix}
&&\\ 
&\Leftrightarrow \quad \begin{pmatrix}u \\ v \\ t\end{pmatrix} = \begin{pmatrix}
(-1)^n-(-2)^n \\
(-2)^n \\
(-1)^n-(-2)^n
\end{pmatrix}
\end{flalign*}

\vspace{5mm}

\noindent En conséquence :

\vspace{-10mm}

\[
\boxed{
\forall n \in \mathbb{Z} \quad f^n(0, 1, 0) = ((-1)^n-(-2)^n, (-2)^n, (-1)^n-(-2)^n)
}
\]


\end{document}

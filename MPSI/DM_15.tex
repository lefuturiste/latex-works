\documentclass[11pt]{article}
\usepackage[english]{babel}
\usepackage[utf8]{inputenc}
\usepackage{fancyhdr}
\usepackage{amsmath}
\usepackage{amssymb}
\usepackage{amsfonts}
\usepackage{mathrsfs}
\usepackage{mathtools}
\usepackage{indentfirst}
\usepackage{hyperref}
\usepackage{tikz,amsmath}
\usepackage{tikz,tkz-tab,amsmath}
\usetikzlibrary{trees}
\usetikzlibrary{arrows}
\usepackage{cancel}
\usepackage{xfrac}  
\usepackage{stmaryrd}

\usepackage{minted}
\usepackage{lmodern}

\usepackage{enumerate}

%\setlength{\TPHorizModule}{1mm}
%\setlength{\TPVertModule}{1mm}
  
% to disable automat²ic indentation on new paragraphs
%\setlength{\parindent}{0pt}

\hypersetup{
    colorlinks=true,
    linkcolor=blue,
    filecolor=magenta,
    urlcolor=blue,
}
\urlstyle{same}

\parskip 1ex

\usepackage{pythonhighlight}

\usepackage{geometry}
 \geometry{
 a4paper,
 left=20mm,
 top=20mm,
 bottom=20mm,
 right=20mm
 }
\usepackage{multicol}
\pagestyle{fancy}
\fancyhf{}
\lhead{Matthieu Bessat}
\rhead{DM 15 - 1 mars. 2021}
\rfoot{Page \thepage}

\makeatletter
\def\@seccntformat#1{%
  \expandafter\ifx\csname c@#1\endcsname\c@section\else
  \csname the#1\endcsname\quad
  \fi}
\makeatother

\makeatletter
\renewcommand*\env@matrix[1][*\c@MaxMatrixCols c]{%
  \hskip -\arraycolsep
  \let\@ifnextchar\new@ifnextchar
  \array{#1}}
\makeatother

\newcommand{\vspacem}{\vspace{3mm}}
\newcommand{\bfrac}[2]{\displaystyle\frac{#1}{#2}}
\newcommand{\blim}[1]{\displaystyle\lim_{#1}}
\newcommand{\bbinom}[2]{\displaystyle\binom{#1}{#2}}
\newcommand{\N}{\mathbb{N}}
\newcommand{\R}{\mathbb{R}}
\newcommand{\Imag}{\text{Im}}
\newcommand{\cotan}{\text{cotan}}

\newcommand\aug{\fboxsep=-\fboxrule\!\!\!\fbox{\strut}\!\!\!}
% just deleted a macro
%
%
%

\begin{document}
\enskip

\vspace{-10px}
\begin{center}
\fontsize{20pt}{20pt}\selectfont
\textbf{DM 15}
\end{center}
\fontsize{11pt}{11pt}\selectfont

\vspace{15px}

\setlength{\abovedisplayskip}{0pt}
\setlength{\belowdisplayskip}{0pt}
\setlength{\abovedisplayshortskip}{0pt}
\setlength{\belowdisplayshortskip}{0pt}

\textbf{1.a.}

\vspace{2px}

\begin{flalign*}
\sin((2n+1)t) &= \Imag(e^{i(2n+1)t} = \Imag\Big((e^{it})^{2n+1}\Big) = \Imag\Big((\cos t + i\sin t)^{2n+1}\Big) &&\\
&= \Imag\Bigg(\sum_{j=0}^{2n+1} \binom{2n+1}{j} \cos^{2n+1-j}(t) (i \sin(t))^j \Bigg)  \quad\text{(D'après le binôme de Newton)} &&\\
&= \sum_{j=0}^{2n+1} \binom{2n+1}{j} \cos^{2n+1-j}(t) \,\Imag(i^j) \sin^j(t) &&\\
\end{flalign*}

\vspace{-5px}
\noindent On effectue alors un changement d'indice, en posant $j=2k+1$ on obtient :

\[
\sin((2n+1)t) = \sum_{k=0}^{n} \binom{2n+1}{2k+1} \cos^{2n+1-(2k+1)}(t) \,\Imag(i^{2k+1}) \sin^{2k+1}(t)
\]

\vspace{7px}

\noindent Or on a, \;\; $\Imag(i^{2k+1}) = \Imag(i^{2k}\times i) = i^{2k} = (i^2)^k = (-1)^k$ \;\; ce qui donne :

\[
\sin((2n+1)t) = \sum_{k=0}^{n} \binom{2n+1}{2k+1} \cos^{2(n-k)}(t) \,(-1)^k \sin^{2k+1}(t)
\]

\vspace{20px}

\noindent Ainsi :

\vspace{-37px}

\[\boxed{\; \forall n \in \mathbb{N} \text{ et } t \in \mathbb{R}, \quad \sin((2n+1)t) = \sum_{k = 0}^n{\binom{2n+1}{2k+1}(-1)^k\sin^{2k+1}(t) \cos^{2(n-k)}(t)} \;} \]

\vspace{10px}

\textbf{1.b.}

\begin{flalign*}
\bfrac{\sin((2n+1)t)}{\sin^{2n+1}(t)} &= \sum_{k=0}^{n} \binom{2n+1}{2k+1} \, (-1)^k \, \bfrac{\sin^{2k+1}(t)}{\sin^{2n+1}(t)} \cos^{2(n-k)}(t) &&\\
&= \sum_{k=0}^{n} \binom{2n+1}{2k+1} \, (-1)^k   \sin^{2(k-n)}(t) \cos^{2(n-k)}(t) &&\\
&= \sum_{k=0}^{n} \binom{2n+1}{2k+1} \, (-1)^k  \bigg(\frac{\cos t}{ \sin
t}\bigg)^{2(n-k)} &&\\
&= \sum_{k=0}^{n} \binom{2n+1}{2k+1} \, (-1)^k  (\cotan^2(t))^{n-k} &&\\
\end{flalign*}

\vspace{8px}

\noindent Ainsi :

\vspace{-28px}

\[\boxed{\; \forall n \in \mathbb{N}, \quad \exists \; P_n \in \R[X] \text{ vérifiant } \bigstar \;} \]

\vspace{10px}

\textbf{1.c.}

\vspace{5px}

\noindent \underline{Raisonnons par l'absurde :}

\noindent Supposons qu'il existe deux polynômes distincts $P_n$ et $P$ vérifiant $\bigstar$ 

\noindent Vu que la fonction $\cotan$ réalise une bijection de $]0, \tfrac{pi}{2}[$ sur $\R^*_+$.

\noindent On a donc $P - P_n = 0$, le polynôme $P - P_n$ possède une infinité de racine, c'est donc le polynôme nul.

\noindent Ainsi $P = P_n$ ce qui est contradictoire. On déduit que : \boxed{ \forall n \in \N, \;\; \text{$P_n$ est l'unique polynôme vérifiant $\bigstar$}}

\vspace{10px}

\textbf{1.d.}

\vspace{5px}

\noindent On a $P_n(\cotan^2(t)) = \bfrac{\sin((2n+1)t)}{\sin^{2n+1}(t)}$

\noindent Ainsi $P_n$ s'annule quand $\sin((2n+1)t) = 0$

\noindent Donc $P_n$ s'annule quand $t$ vérifie $(2n+1)t = k \pi$ avec $k \in \llbracket 1,n \rrbracket$. On a donc $t = \bfrac{k\pi}{2n+1}$

\vspace{20px}

\noindent Ainsi :

\vspace{-37px}

\[\boxed{\; \text{Les racines de $P_n$ sont pour $k \in \llbracket 1,n \rrbracket, \quad x_k = \cotan^2\Bigg(\bfrac{k\pi}{2n+1}\Bigg)$} \;} \]

\vspace{10px}

\textbf{2.a.} Procédons par récurrence.

\vspace{5px}

\noindent \textbf{\underline{Initialisation} :} \, Pour $n = 1$,

\noindent On a $Q_1$, polynôme de degré 1 qui s'écrit : $Q_1 = a_1(X-x_1) = a_1 X + a_0$

\noindent $Q_1$ possède une seule racine $x_1$, et on a :

$a_1(X-x_1) = a_1 X + a_0 \Leftrightarrow a_1X-a_1x_1 = a_1 X + a_0 \Leftrightarrow -a_1 x_1 = a_0 \Leftrightarrow x_1 = -\bfrac{a_0}{a_1}$

\noindent Ainsi la propriété est vérifié au rang $n = 1$.

\vspace{5px}

\noindent \textbf{\underline{Hérédité} :} \, On suppose la propriété vrai pour un certain rang $n \in \N^*$. Montrons vrai la propriété au rang $n+1$.

$T = \displaystyle\sum_{k=1}^{n+1} a_i X^i = a_{n+1} \prod^{n+1}_{k=1}(X-x_k) = a_{n+1}(X-x_{n+1})\prod^n_{k=1} (X-x_k)$

Au rang $n$ on a : $\displaystyle\sum_{k=1}^n(X-x_k) = \bfrac{1}{a_n}\sum_{i=0}^na_i X^i$ donc :

$T = a_{n+1}(X-x_{n+1})\bfrac{1}{a_n}\sum_{i=0}^na_i X^i$

$T = a_{n+1}(X-x_{n+1})(\bfrac{1}{a_n} a_n X^n + \bfrac{1}{a_n} a_{n-1} X^{n-1} + \text{Reste})$

\vspace{5px}
\noindent D'après l'hypothèse de récurrence :

\vspace{5px}
$T = a_{n+1}(X-x_{n+1})(X^n - \displaystyle\sum_{k=1}^n x_k X^{n-1} + \text{Reste})$

\vspace{2px}
\noindent Donc après on développe et on utilise l'identification pour montrer que : (j'ai l'impression qu'il faut faire ça)

\vspace{5px}
$\displaystyle\sum_{k=1}^n x_k = -\bfrac{a_{n-1}}{a_n}$

\vspace{5px}

\noindent\textbf{Comment faire ?? Je sais pas}

\vspace{5px}

\vspace{5px}

\noindent \textbf{\underline{Conclusion} :} \, La propriété est vrai au rang $n = 1$ et le passage du rang $n$ au rang $n+1$ est possible, ainsi d'après le principe de récurrence, la propriété suivante est vrai pour tout $n \in \N^*$ :

Si Q est un polynôme de degré n s'écrivant :
$Q = a_n \displaystyle\prod^n_{k=1}(X-x_k) = \sum^n_{i=0} a_i X^i$.

\vspace{20px}

Alors on a :

\vspace{-37px}

\[\boxed{\; \sum^n_{k=1} x_k = -\bfrac{a_{n-1}}{a_n} \;}\]

\vspace{10px}

\textbf{2.b.}

\vspace{5px}

\noindent Par identification on trouve $a_n$ et $a_{n-1}$ :

$a_n = \binom{2n+1}{2\times0 +1}(-1)^0 = \binom{2n+1}{1}$
$a_{n-1} = \binom{2n+1}{2\times1+1}(-1)^1 = -\binom{2n+1}{3}$

\vspace{5px}

\noindent Ainsi :

\begin{flalign*}
-\bfrac{a_{n-1}}{a_n} &= \bfrac{\binom{2+n1}{3}}{\binom{2n+1}{1}} = \bfrac{(2n+1)!}{3! (2n-2)!}\times\bfrac{1!(2n)!}{(2n+1)!} &&\\
&= \bfrac{(2n)!}{3!(2n-2)!} = \bfrac{(2n-2)!\times(2n-1)\times(2n)}{3! (2n-2)!} &&\\
&= \bfrac{n(2n-1)}{3}
\end{flalign*}

\vspace{5px}

\[\boxed{\; \text{La somme des racines de $P_n$ est : } S_n = \bfrac{n(2n-1)}{3} \;} \]

\vspace{10px}

\textbf{3.a.}

\begin{itemize}
  \item {
    Montrons $\sin(t) \leq t$ c'est-à-dire montrons $\forall t \in [0, \tfrac{\pi}{2}[, \quad t-\sin(t) \geq 0$.
    
    $(t-\sin t)' = 1 - \cos t$
    
    Or $\cos t \in [-1, 1]$ donc l'expression $1-\cos t$ est positive et donc c est croissant.
    
    Or en évaluant l'expression $1-\sin t$ en 0, on obtient $1-\sin 0 = 0$. $1- \sin t$ est donc positive.
    
    Ainsi on a $\forall t \in [0, \tfrac{\pi}{2}[, \quad \sin(t) \leq t$
  }
  \item {
    De la même manière montrons $\forall t \in [0, \tfrac{\pi}{2}], t \leq \tan t$, on étudie le signe de $\tan t - t$ :
    
    $(\tan t - t)' = 1 + \tan^2(t) -1 = \tan^2(t)$. Donc $(\tan t -t)' \geq 0$.
    
    Ainsi $\tan t - t$ est croissante.
    
    Or vu que $\tan 0 - 0 = 0$, on a $\tan t - t \geq > 0$, donc $t \leq \tan t$
  }
\end{itemize}

\noindent Nous avons d'une part $\sin t \leq t$ et d'une autre $t \leq \tan t$. Ainsi :

\vspace{5px}

\[\boxed{\; \forall t \in [0, \tfrac{\pi}{2}[, \quad \sin t \leq t \leq \tan t \;} \]

\vspace{10px}

\noindent On a d'une part :

\begin{flalign*}
& \sin t \leq t \Longleftrightarrow \sin^2(t) \leq t^2 &&\\
&\Longleftrightarrow \bfrac{\sin^2(t)}{\cos^2(t) + \sin^2(t)} \leq t^2 \Longleftrightarrow \bfrac{\cos^2(t) + \sin^2(t)}{\sin^2(t)} \geq \bfrac{1}{t^2} \Longleftrightarrow \cotan^2(t) + 1 \geq \bfrac{1}{t^2} &&\\
\end{flalign*}

\noindent Et de l'autre :

\begin{flalign*}
t \leq \tan t \Longleftrightarrow \bfrac{1}{t^2} \geq \bfrac{1}{\tan t} \Longleftrightarrow \bfrac{1}{t^2} \geq \cotan^2(t)
\end{flalign*}

\vspace{5px}

\noindent Ainsi, $\cotan^2(t) + 1 \geq \bfrac{1}{t^2} \geq \cotan^2(t)$, où encore :

\vspace{5px}

\[\boxed{\; \forall t \in ]0, \tfrac{\pi}{2}[, \quad \cotan^2(t) \leq \bfrac{1}{t^2} \leq 1+\cotan^2(t) \;} \]

\vspace{10px}

\newpage

\textbf{3.b.}

\noindent On évalue l'inégalité obtenue à la question 3.a. en $t = \bfrac{k\pi}{2n+1}$, on obtient alors :

\begin{center}
$\cotan^2\Bigg(\bfrac{k\pi}{2n+1}\Bigg) \leq \bfrac{1}{\big(\frac{k\pi}{2n+1}\big)^2} \leq 1+\cotan^2\Bigg(\bfrac{k\pi}{2n+1}\Bigg)$

$\Longleftrightarrow \cotan^2\Bigg(\bfrac{k\pi}{2n+1}\Bigg) \leq \bfrac{(2n+1)^2}{k^2\pi^2} \leq 1+\cotan^2\Bigg(\bfrac{k\pi}{2n+1}\Bigg)$

$\Longleftrightarrow \displaystyle\sum_{k=1}^n x_k \leq \displaystyle\sum_{k=1}^n \bfrac{(2n+1)^2}{\pi^2 k^2} \leq \sum_{k=1}^n n + \sum_{k=1}^n x_k$

\vspace{5px}

D'après la question 2.b. on connaît la valeur de $\sum_{k=1}^n x_k$ :

\vspace{5px}

$\Longleftrightarrow \bfrac{n(2n-1)}{3} \leq \sum_{k=1}^n \bfrac{(2n+1)^2}{\pi^2 k^2} \leq n + \bfrac{n(2n-1)}{3}$
\end{center}

\[\boxed{\; \forall n \in \N^*, \quad \bfrac{n(2n-1)}{3} \leq \sum_{k=1}^n \bfrac{(2n+1)^2}{\pi^2 k^2} \leq n + \bfrac{n(2n-1)}{3} \;}\]

\vspace{10px}

\textbf{3.c.}

\noindent On note, pour tout $n \in \N^*$, la suite $(u_n)$ tel que $u_n = \displaystyle\sum_{k=1}^n \bfrac{1}{k^2}$

\noindent D'après la question 3.b on a un encadrement de la suite $(u_n)$ :

$\forall n \in \N^*, \quad \bfrac{n(2n-1)}{3}\times \bfrac{\pi^2}{(2n+1)^2} \geq \sum_{k=1}^n \frac{1}{k^2} \geq \bfrac{\pi^2}{(2n+1)^2} \Bigg( n + \bfrac{n(2n-1)}{3} \Bigg)$

\vspace{5px}

\noindent On étudie séparément les limites de la suite de chaque borne :

\begin{itemize}
  \item {
    $\bfrac{\pi^2 n(2n-1)}{3(2n+1)^2} = \bfrac{\pi^2}{3} \times \bfrac{2n^2-n}{4n^2 + 4n + 1}$
    $=  \bfrac{\pi^2}{3} \times \bfrac{2-\tfrac{1}{n}}{4+\tfrac{4}{n}+\tfrac{1}{n^2}} \longrightarrow \bfrac{\pi^2}{3} \times \bfrac{2}{4} = \bfrac{\pi^2}{6}$
  }
  \item {
    $\bfrac{\pi^2}{(2n+1)^2} \Bigg( n + \bfrac{n(2n-1)}{3} \Bigg) = \bfrac{3n\pi^2 + \pi^2 n(2n-1)}{3(2n+1)^2}$
    
    $= \bfrac{\pi^2}{3} \times \bfrac{n(2n+2)}{(2x+1)^2} = \bfrac{\pi^2}{3} \times \bfrac{n^2+2x}{4n^2+4n+1} = \bfrac{\pi^2}{3} \times \bfrac{2+\tfrac{2}{n}}{4+\tfrac{4}{n}+\tfrac{1}{n^2}}$
    $=  \longrightarrow \bfrac{\pi^2}{3} \times \bfrac{2}{4} = \bfrac{\pi^2}{6}$
  }
\end{itemize}

\noindent Les deux bornes convergent vers la même limite, ainsi d'après le théorème des gendarmes :

\[\boxed{\; \text{La suite $(u_n)$ converge et sa limite est $\bfrac{\pi^2}{6}$} \;}\]
\end{document}


\documentclass{article}
\usepackage[english]{babel}
\usepackage[utf8]{inputenc}
\usepackage{fancyhdr}
\usepackage{amsmath}
\usepackage{amsfonts}
\usepackage{mathrsfs}
\usepackage{mathtools}
\usepackage{indentfirst}
\usepackage{hyperref}
\usepackage{tikz,amsmath}
\usepackage{tikz,tkz-tab,amsmath}
\usetikzlibrary{trees}
\usetikzlibrary{arrows}
\usepackage{enumerate}

%\setlength{\TPHorizModule}{1mm}
%\setlength{\TPVertModule}{1mm}
  
\hypersetup{
    colorlinks=true,
    linkcolor=blue,
    filecolor=magenta,
    urlcolor=blue,
}
\urlstyle{same}

\parskip 1ex

\usepackage{geometry}
 \geometry{
 a4paper,
 left=20mm,
 top=20mm,
 bottom=20mm,
 right=20mm
 }

\pagestyle{fancy}
\fancyhf{}
\lhead{Matthieu Bessat}
\rhead{DM 1}
\rfoot{Page \thepage}

\makeatletter
\def\@seccntformat#1{%
  \expandafter\ifx\csname c@#1\endcsname\c@section\else
  \csname the#1\endcsname\quad
  \fi}
\makeatother

\newcommand{\vspacem}{\vspace{2mm}}
\newcommand{\bfrac}[2]{\displaystyle\frac{#1}{#2}}
\newcommand{\bbinom}[2]{\displaystyle\binom{#1}{#2}}

\newif\ifquoteopen
\catcode`\"=\active % lets you define `"` as a macro
\DeclareRobustCommand*{"}{%
   \ifquoteopen
     \quoteopenfalse ''%
   \else
     \quoteopentrue ``%
   \fi
}

\begin{document}

\subsection*{Coefficients binomiaux}

\textbf{1.a}
\vspacem

\renewcommand{\theenumi}{(\roman{enumi})}%
\begin{enumerate}
  \item $\bbinom{10}{3} = 120$
  \item $\bbinom{10}{4} = 210$
  \item $\bbinom{10}{4} = 210$
  \item $\bbinom{10}{5} = 252$
\end{enumerate}

\textbf{1.b}
\vspacem

\noindent Les 4 cas couvrent toutes les possibilités d'arranger un sous ensemble de $E$ de cardinal $5$

% \renewcommand{\theenumi}{(\roman{enumi})}%
% \begin{enumerate}
%   \item $a$ et $b$ sont fixés 
%   \item $a$ est fixé
%   \item $b$ est fixé
%   \item Rien n'est fixé
% \end{enumerate}

\noindent On a donc : 

\vspacem
$\bbinom{12}{5} = \bbinom{10}{3} + 2\bbinom{10}{4} + \bbinom{10}{5}$

\vspacem
\textbf{1.c}
\vspacem

On considère un ensemble $E_n$ à $n$ éléments.
On veut dénombrer toutes les parties à $p$ éléments de cet ensemble.
On sait que nous avons $\binom{n}{p}$ parties dans cet ensemble $E_n$.

Il existe une autre manière de compter les parties :
On considère deux éléments particuliers $a$ et $b$ appartenant à $E_n$
On compte ensuite toute les parties de à $p$ éléments qui :

\begin{itemize}
  \item Contiennent $a$ et $b$
  \item Contiennent $b$ mais pas $a$
  \item Contiennent $a$ mais pas $b$
  \item Contiennent ni $a$ ni $b$
\end{itemize}

\vspacem
\textbf{1.d}
\vspacem

\noindent\begin{flalign}
\bbinom{n}{p} &= \bbinom{n-1}{p-1} + \bbinom{n-1}{p}  &&\\
&= \bbinom{n-2}{p-1} + \bbinom{n-2}{p-2} + \bbinom{n-1}{p} &&\\
&= \bbinom{n-2}{p-1} + \bbinom{n-2}{p-2} +
\bbinom{n-2}{p-1} + \bbinom{n-2}{p} &&\\
&= \bbinom{n-2}{p-2} +  2 \bbinom{n-2}{p-1} + \bbinom{n-2}{p}
\end{flalign}

\textbf{2.}
\vspacem

Si $0 \leq q \leq p \leq n$ : 

\noindent\begin{flalign}
\bbinom{n}{q} \bbinom{n-q}{n-p} &= \bfrac{n!}{q! \, (n-q)!} \times \bfrac{(n-q)!}{(n-p)! \, (n-q-n+p)!} &&\\
&= \bfrac{n!}{q! \, (n-p)! \, (p-q)!} &&\\
&= \bfrac{n!}{p! \, (n-p)!} \times \bfrac{p!}{q! \, (p-q)!} &&\\
&= \bbinom{n}{p} \bbinom{p}{q}
\end{flalign}

\vspace{2mm}
\newpage

\subsection*{Raisonnements}

\textbf{1.a.i}

\noindent On a $x = y = 0$

\noindent Ainsi :

$f(x)f(y) - f(xy) = x + y$

$\Leftrightarrow f(0)f(0)-f(0\times 0) = 0 + 0$

$\Leftrightarrow (f(0))^2 - f(0) = 0$

$\Leftrightarrow (f(0))^2 = f(0)$

\noindent Donc $f(0) = 1$ ou $f(0) = 0$

\vspacem
\textbf{1.a.ii}

$f(x)f(y) - f(xy) = x + y$

$\Leftrightarrow f(1)f(0)-f(1 \times 0) = 1 + 0$

$\Leftrightarrow f(1)f(0)-f(0) = 1$

$\Leftrightarrow f(0)(f(1) - 1) = 1$

\noindent Un produit non nul a nécessairement deux facteurs non nul $f(0)$ et $(f(1) - 1)$

\noindent Donc, d'après 1.a.(i), $f(0) = 1$, ainsi :

$\Leftrightarrow f(1) - 1 = 1$

$\Leftrightarrow f(1) = 2$

\noindent Quand $x = 1$ et $y = 0$, $f(0) = 1$ et $f(1) = 2$

\vspacem
\textbf{1.a.iii}

$f(x)f(y) - f(xy) = x + y$

$\Leftrightarrow f(x)f(0)-f(0) = x$

$\Leftrightarrow f(0)(f(x)-1) = x$

\noindent D'après 1.a.(ii), $f(0) = 1$ :

$\Leftrightarrow 1 \times (f(x)-1) = x$

$\Leftrightarrow f(x) = x + 1$

\vspacem
\textbf{1.b}

\noindent Avec $f(x) = x+1$, on a :

$f(x)f(y) - f(xy) = x + y$

$\Leftrightarrow (x+1)(y+1) - (xy + 1) = x + y$

$\Leftrightarrow xy + x + y + 1 - xy - 1 = x + y$

$\Leftrightarrow x + y = x + y$

\newpage
\vspacem
\textbf{2.}

\noindent On étudie le signe de l'expression $x^2 - x + 1 - |x-1|$ pour $x \geq 0$ et $x \leq 0$.

\noindent On procède par disjonction de cas :

\noindent \underline{Si $x \geq 1$}:

Alors $x-1 \geq 0$, donc $|x-1| = x-1$

$x^2-x+1-|x-1|$

$\Leftrightarrow x^2 - x + 1 - x + 1$

$\Leftrightarrow x^2 - 2x + 2$

On calcule la dérivé pour avoir les variations de $x^2-2x+2$

$(x^2 - 2x + 2)' = 2x - 2$

$2x-2 \geq 0 \Leftrightarrow 2x \geq 2 \Leftrightarrow x \geq 1$

Donc $\forall x \in [1, +\infty[$, $|x-1| \leq x^2-x+1$

% $\Delta = (-2)^2 - 4\times1\times2 = 4 - 8 = -4$

% Vu que $\Delta < 0$, la fonction n'a pas de racines. De plus, le coefficient de $x^2$ est positif. Donc 

\noindent Si \underline{$x \leq 1$}:

Alors $x -1 \leq 0$, donc $|x-1| = -(x-1) = -x+1$

$x^2-x+1 - ( -x + 1)$

$\Leftrightarrow x^2 - x + 1 + x - 1$

$\Leftrightarrow x^2$

On a $\forall x \geq 1, \, x^2 \geq 0$

Donc $\forall x \in ]-\infty, \, 1]$, $|x-1| \leq x^2-x+1$


% $|x-1| \leq x^2 - x + 1$
% $\Leftrightarrow |x-1| \leq x^2 - (x+1)$

% On étudie les variations de $|x-1|$ et de $x^2-x+1$

% $(x^2-x+1)' = 2x - 1$

% $2x-1 > 0 \Leftrightarrow 2x > 1 \Leftrightarrow x > \frac{1}{2}$

% \begin{tikzpicture}
%   \tkzTabInit[lgt=2.2, espcl=3]{
%     $x$ /1.5,
%     $2x-1$ /1.5,
%     $x^2 - x + 1$ /1.5,
%     $|x-1|$ /1.5
%   }{
%     $-\infty$,
%     $0$,
%     $\bfrac{1}{2}$,
%     $1$,
%     $+\infty$
%   }%
%   \tkzTabLine{,,-,,z,,+}
%   \tkzTabVar{
%     +/,
%     R/,
%     -/$\frac{3}{4}$,
%     R/,
%     +/,
%   }
%   \tkzTabIma{3}{1}{2}{$1$}
%   \tkzTabVar{
%     +/,
%     R/,
%     R/,
%     -/$0$,
%     +/
%   }
%   \tkzTabIma{4}{1}{2}{$1$}
%   \tkzTabIma{4}{1}{3}{$\frac{1}{2}$}
% %   \tkzTabVar{
% %     -/,
% %     -/,
    
% %     +/$\displaystyle\frac{2}{3}\sqrt{\displaystyle\frac{1}{3}}$,
% %     -/
% %   }
% \end{tikzpicture}

\noindent Par réunion, $\forall x \in \mathbb{R},\, |x-1| \leq x^2-x+1$

\vspacem
\textbf{3.}

\noindent $n = a^2 + b^2$

\noindent Pour chaque valeur possible de $a \mod 4$, $b \mod 4$, $a^2 \mod 4$, $b^2 \mod 4$ on regarde les valeurs possibles de $(a^2 + b^2) \mod 4$ grâce à un tableau.

\renewcommand{\arraystretch}{1.5}
\begin{center}
\begin{tabular}{ |c | c | c | c | c | c | }
 \hline
 $a \mod 4$ & $b \mod 4$ & $a^2 \mod 4$ & $b^2 \mod 4$ & $(a^2 + b^2) \mod 4$ \\
 \hline
 0 & 0 & 0 & 0 & 0 \\
 0 & 1 & 0 & 1 & 1 \\  
 0 & 2 & 0 & 0 & 0 \\  
 0 & 3 & 0 & 1 & 1 \\
 \hline
 1 & 0 & 1 & 0 & 1 \\
 1 & 1 & 1 & 1 & 2 \\
 1 & 2 & 1 & 0 & 1 \\
 1 & 3 & 1 & 1 & 2 \\
 \hline
 2 & 0 & 0 & 0 & 0 \\
 2 & 1 & 0 & 1 & 1 \\
 2 & 2 & 0 & 0 & 0 \\
 2 & 3 & 0 & 1 & 1 \\
 \hline
 3 & 0 & 1 & 0 & 1 \\
 3 & 1 & 1 & 1 & 2 \\
 3 & 2 & 1 & 0 & 1 \\
 3 & 3 & 1 & 1 & 2 \\
 \hline
\end{tabular}
\end{center}

\noindent D'après ce tableau, il n'y a pas de reste de la division de $n$ par $4$ valant 3. Ainsi la proposition est vraie.

\end{document}

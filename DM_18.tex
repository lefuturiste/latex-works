\documentclass{article}
\usepackage[english]{babel}
\usepackage[utf8]{inputenc}
\usepackage{fancyhdr}
\usepackage{amsmath}
\usepackage{amsfonts}
\usepackage{mathrsfs}
\usepackage{mathtools}
\usepackage{indentfirst}
\usepackage{hyperref}
\usepackage{tikz,tkz-tab,amsmath}
\usetikzlibrary{trees}
\usepackage{minted}

\hypersetup{
    colorlinks=true,
    linkcolor=blue,
    filecolor=magenta,      
    urlcolor=blue,
}
\urlstyle{same}

\parskip 1ex

\usepackage{geometry}
 \geometry{
 a4paper,
 left=20mm,
 top=20mm,
 bottom=20mm,
 right=20mm
 }

\pagestyle{fancy}
\fancyhf{}
\rhead{Corrigé DM 18}
\lhead{Ex 51 p.208}
\rfoot{Page \thepage}

\makeatletter
\def\@seccntformat#1{%
  \expandafter\ifx\csname c@#1\endcsname\c@section\else
  \csname the#1\endcsname\quad
  \fi}
\makeatother

\newcommand{\vspacem}{\vspace{2mm}}
\newcommand{\bfrac}[2]{\displaystyle\frac{#1}{#2}}

\begin{document}

\tikzstyle{level 1}=[level distance=3.5cm, sibling distance=3.5cm]
\tikzstyle{level 2}=[level distance=3.5cm, sibling distance=2cm]
\tikzstyle{bag} = [text width=4em, text centered]
\tikzstyle{end} = [circle, minimum width=3pt,fill, inner sep=0pt]
\begin{tikzpicture}[grow=right, sloped]
\node[bag] {\space}
    child {
        node[bag] {$N_1$}        
            child {
                node[end, label=right:
                    {$N_2$}] {}
                edge from parent
                node[rotate=16, midway,fill=white]  {$\bfrac{1}{5}$}
            }
            child {
                node[end, label=right:
                    {$R_2$}] {}
                edge from parent
                node[rotate=-15, midway,fill=white]  {$\bfrac{4}{5}$}
            }
            edge from parent 
            node[rotate=26, midway,fill=white]  {$\bfrac{1}{3}$}
    }
    child {
        node[bag] {$R_1$}        
        child {
                node[end, label=right:
                    {$N_2$}] {}
                edge from parent
                node[rotate=16, midway,fill=white]  {$\bfrac{1}{3}$}
            }
            child {
                node[end, label=right:
                    {$R_2$}] {}
                edge from parent
                node[rotate=-15, midway,fill=white]  {$\bfrac{2}{3}$}
            }
        edge from parent         
            node[rotate=-26, midway,fill=white]  {$\bfrac{2}{3}$}
    };
\end{tikzpicture}

\textbf{1.a}
\vspacem

\noindent $P(R_1 \cap R_2) = P(R_1) \times P_{R_1}(R_2) = \bfrac{4}{6} \times \bfrac{4}{6} = \bfrac{4}{9}$

\noindent La probabilité que les boules tirées soient rouges est $\bfrac{4}{9}$.

\vspacem
\textbf{1.b}
\vspacem

\noindent D'après la formule des probabilités totales :

$P(N_2) = P(R_1 \cap N_2) + P(N_1 \cap N_2) = \bfrac{2}{3} \times \bfrac{1}{3} + \bfrac{1}{3} \times \bfrac{1}{5} = \bfrac{13}{45}$

\noindent La probabilité que la seconde boule tiré soit noire est $\bfrac{13}{45}$.

\noindent $P_{N_2}(R_1) = \bfrac{P(R_1 \cap N_2)}{P(N_2)} = \bfrac{\frac{2}{3} \times \frac{1}{3}}{\frac{13}{45}} = \bfrac{10}{13}$

\noindent La probabilité que la première boule soit rouge sachant que la seconde est noire est $\bfrac{10}{13}$.

\vspacem
\textbf{2.a}
\vspacem

\noindent $p$ correspond à la probabilité de tirer une boule rouge pendant un tirage. On sait qu'il y a $n$ boules noires et 4 autres boules rouges. Le nombre total de balle est $n+4$ donc $p = \bfrac{4}{n+4}$.

\vspacem
\textbf{2.b}
\vspacem

\noindent $q_n$ correspond à la probabilité de l'évènement "tirer une boule noire ou plus pendant les quatre tirages". L'évènement contraire de probabilité $r_n$ est "tirer quatre boules rouges". Donc :

$q_n = 1 - r_n$

\noindent Or, la probabilité de tirer quatre boules rouges, on peut la connaître grâce à loi binomiale donnée, c'est $p(X = 4)$:

$p(X = 4) = \displaystyle{4 \choose 4}\bigg(\bfrac{4}{n+4}\bigg)^4 q^{4-4}$

$p(X = 4) = \displaystyle{4 \choose 4}\bigg(\bfrac{4}{n+4}\bigg)^4 \times 1$

$p(X = 4) = 1 \times \bigg(\bfrac{4}{n+4}\bigg)^4$

$r_n = p(X = 4) = \bigg(\bfrac{4}{n+4}\bigg)^4$

\noindent Donc:

$q_n = 1 - \bigg(\bfrac{4}{n+4}\bigg)^4$

\newpage
\vspacem
\textbf{2.c}
\vspacem

\noindent Pour répondre à cette question on peux utiliser un algorithme en python par exemple :

\inputminted{python}{DM_18_python.py}

\noindent Si on exécute la fonction avec $\text{valeur} = 0,9999$, on obtient $n = 36$. Donc, afin que la probabilité de l'évènement "avoir au moins une boule tirée sur quatre noire" soit supérieur à 0,9999 il faut avoir au moins 36 boules noires dans l'urne. De manière générale plus on a de boules noire moins on a de chance de tirer une boule rouge et donc $q_n$ tend vers 1.

\end{document}

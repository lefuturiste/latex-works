\documentclass{article}
\usepackage[english]{babel}
\usepackage[utf8]{inputenc}
\usepackage{fancyhdr}
\usepackage{amsmath}
\usepackage{amsfonts}
\usepackage{mathrsfs}
\usepackage{mathtools}
\usepackage{indentfirst}
\usepackage{hyperref}
\usepackage{tikz,tkz-tab,amsmath}
\hypersetup{
    colorlinks=true,
    linkcolor=blue,
    filecolor=magenta,      
    urlcolor=blue,
}
\urlstyle{same}

\usepackage{scalerel,stackengine}
\stackMath
\newcommand\reallywidehat[1]{%
\savestack{\tmpbox}{\stretchto{%
  \scaleto{%
    \scalerel*[\widthof{\ensuremath{#1}}]{\kern-.6pt\bigwedge\kern-.6pt}%
    {\rule[-\textheight/2]{1ex}{\textheight}}%WIDTH-LIMITED BIG WEDGE
  }{\textheight}% 
}{0.5ex}}%
\stackon[1pt]{#1}{\tmpbox}%
}
\parskip 1ex

\usepackage{geometry}
 \geometry{
 a4paper,
 left=20mm,
 top=20mm,
 bottom=20mm,
 right=20mm
 }

\pagestyle{fancy}
\fancyhf{}
\rhead{Correction}
\lhead{DC8}
\rfoot{Page \thepage}


\makeatletter
\def\@seccntformat#1{%
  \expandafter\ifx\csname c@#1\endcsname\c@section\else
  \csname the#1\endcsname\quad
  \fi}
\makeatother

\begin{document}

\textbf{1.a}

\vspace{2mm}

\noindent 

$\overrightarrow{AB}
\begin{pmatrix}
   x_B-x_A\\
   y_B-y_A\\
   z_B-z_A
\end{pmatrix}
=
\begin{pmatrix}
   1+1\\
   2-2\\
   4-0
\end{pmatrix}
= 
\begin{pmatrix}
   2\\
   0\\
   4
\end{pmatrix}
$

\vspace{2mm}

$\overrightarrow{AC}
\begin{pmatrix}
   x_C-x_A\\
   y_C-y_A\\
   z_C-z_A
\end{pmatrix}
=
\begin{pmatrix}
   -1+1\\
   1-2\\
   1-4
\end{pmatrix}
= 
\begin{pmatrix}
   0\\
   -1\\
   1
\end{pmatrix}
$
\vspace{2mm}

\noindent Il est impossible que les vecteurs $\overrightarrow{AB}$ et $\overrightarrow{AC}$ soient colinéaires : par exemple l'abscisse de $\overrightarrow{AC}$ est nulle, si les deux vecteurs étaient colinéaires, (donc ont leurs coordonnées proportionnelles) alors l'abscisse de $\overrightarrow{AB}$ devrait être aussi nulle. 

\vspace{2mm}

\noindent Les deux vecteurs ne sont pas colinéaires car leurs coordonnées ne sont pas proportionnelles donc les points A, B, et C ne sont pas alignés.

\vspace{2mm}

\textbf{1.b}



\vspace{2mm}

\noindent $\overrightarrow{AB} \cdot \overrightarrow{AC} = x_{AB} \times x_{AC} + y_{AB} \times y_{AC} + z_{AB} \times z_{AC} = 2\times0 + 0\times-1 + 4\times1 = 0 + 0 + 4 = 4$

\vspace{2mm}

\textbf{1.c}

\vspace{2mm}

\noindent $||\overrightarrow{AB}|| = \displaystyle\sqrt{x_{\overrightarrow{AB}}^2 + y_{\overrightarrow{AB}}^2 + z_{\overrightarrow{AB}}^2} = \sqrt{2^2+0+4^2} = \sqrt{4+16} = \sqrt{20} = 2\sqrt{5}$

\vspace{2mm}

\noindent $||\overrightarrow{AC}|| = \displaystyle\sqrt{x_{\overrightarrow{AC}}^2 + y_{\overrightarrow{AC}}^2 + z_{\overrightarrow{AC}}^2} = \sqrt{0+(-1)^2+1^2} = \sqrt{2}$

\vspace{2mm}
\noindent $\overrightarrow{AB} \cdot \overrightarrow{AC} = ||\overrightarrow{AB}|| \times ||\overrightarrow{AC}|| \times \cos\Big(\overrightarrow{AB}, \overrightarrow{AC}\Big)$

\vspace{2mm}

$\Leftrightarrow 4 = 2\sqrt{5} \times \sqrt{2} \times \cos\Big(\overrightarrow{AB}, \overrightarrow{AC}\Big)$

\vspace{2mm}

$\Leftrightarrow \cos\Big(\overrightarrow{AB}, \overrightarrow{AC}\Big) = \displaystyle\frac{4}{2\times \sqrt{5} \times \sqrt{2}} = \displaystyle\frac{2}{\sqrt{10}}$

\vspace{2mm}

$\Leftrightarrow \Big(\overrightarrow{AB}, \overrightarrow{AC}\Big) = \arccos\Bigg({\displaystyle\frac{2}{\sqrt{10}}}\Bigg)$

\vspace{2mm}

\noindent $\reallywidehat{BAC} \simeq 51 \deg$


\vspace{2mm}

\textbf{2.a}

\vspace{2mm}

\noindent D'après la question 1.a, $\overrightarrow{AB}$ et $\overrightarrow{AC}$ sont non colinéaires. Ce sont donc des vecteurs directeurs du plan $(ABC)$

\vspace{2mm}
$\overrightarrow{AB} \cdot \overrightarrow{n} = 2 \times 2 -1 \times 0 - 1 \times 4 = 4 - 4 = 0 $ \qquad  Donc $\overrightarrow{AB} \perp \overrightarrow{n}$

\vspace{2mm}

$\overrightarrow{AC} \cdot \overrightarrow{n} = 0 \times 2 - 1 \times - 1 - 1 \times 1 = 1 - 1 = 0$\qquad Donc $\overrightarrow{AC} \perp \overrightarrow{n}$
\vspace{2mm}

\noindent Donc le vecteur $\overrightarrow{n}$ est bien un vecteur normal au plan $(ABC)$ car $\overrightarrow{n}$ est orthogonal à deux vecteurs non colinéaires de $(ABC)$. 

\vspace{2mm}

\textbf{2.b}

\vspace{2mm}

\noindent Une équation du plan $(ABC)$ est $2x-y-z+d=0$ or $A \in (ABC)$ donc :

\vspace{2mm}

$2x_A - y_A - z_A + d = 0$ \qquad avec $A(-1; 2; 0)$

$\Leftrightarrow 2 \times -1 - 2 + d = 0$

$\Leftrightarrow -4 + d=0$

$\Leftrightarrow d = 4$

\vspace{2mm}

\noindent Donc une équation cartésienne du plan $(ABC)$ est : $2x - y - z + 4 = 0$

\vspace{2mm}

\textbf{3.a}

\vspace{2mm}

\noindent Comme $P_2$ est parallèle au plan d'équation $x - 2z + 6 = 0$ alors $P_2$ a pour vecteur normal $\overrightarrow{u} \Big(1, 0, -2\Big)$.

\noindent Alors une équation cartésienne de $P_2$ est : $x - 2z + d = 0$

\vspace{2mm}

\noindent On utilise alors les coordonnées de $D\Big(0, 0, 1\Big)$ appartenant à $\mathscr{P_2}$ pour trouver la valeur de $d$ :

\vspace{2mm}

$0 - 2\times 1 + d = 0$

$\Leftrightarrow -2 + d = 0$

$\Leftrightarrow d = 2$

\vspace{2mm}

\noindent Donc une équation cartésienne du plan $P_2$ est : $x - 2z + 2 = 0$, cette équation peut s'écrire $x = 2z-2$

\vspace{2mm}

\textbf{3.b}

\vspace{2mm}

$\overrightarrow{u} \Big(3, -1, 1\Big)$ est un vecteur normal de $P_1$

\vspace{2mm}

$\overrightarrow{v} \Big(1, 0, -2\Big)$ est un vecteur normal de $P_2$

\vspace{2mm}

Les coordonnées des deux vecteurs ne sont pas proportionnelles (ordonnée non nulle du vecteur $\overrightarrow{u}$). Donc les vecteurs $\overrightarrow{u}$ et $\overrightarrow{v}$ ne sont pas colinéaires. Donc les plans $P_1$ et $P_2$ ne sont pas parallèles et sont donc sécants.

\vspace{2mm}

\textbf{3.c}

\vspace{2mm}

\noindent On fait un système qui réunit les deux équations cartésienne des deux plans. On prends $z = t$ où $t \in \mathbb{R}$ :

\vspace{2mm}

$\left\{\begin{matrix*}[l]
   3x-y+t+3=0\\
   \\
   x = 2t - 2\\
   \\
   z = t
\end{matrix*}\right.
\Leftrightarrow
\left\{\begin{matrix*}[l]
   3(2t-2) - y + t+ 3 = 0\\
   \\
   x=2t-2\\
   \\
   z= t\\
\end{matrix*}\right.
\Leftrightarrow
\left\{\begin{matrix*}[l]
   6t-6-y+t+3=0\\
   \\
   x=2t-2\\
   \\
   z=t
\end{matrix*}\right.
$

\vspace{2mm}

$
\Leftrightarrow
\left\{\begin{matrix*}[l]
   7t-3-y=0\\
   \\
   x=2t-2\\
   \\
   z=t
\end{matrix*}\right.
\Leftrightarrow
\left\{\begin{matrix*}[l]
   x=2t-2\\
   \\
   y=7t-3\\
   \\
   z=t
\end{matrix*}\right., t \in \mathbb{R}
$

\vspace{2mm}

\noindent Donc $\mathscr{D}$ est bien l'intersection des plan $P_1$ et $P_2$.

\vspace{2mm}

\textbf{4}

\vspace{2mm}

On fait un système qui regroupe une équation cartésienne du plan $(ABC)$ avec la représentation paramétrique de $\mathscr{D}$ :

\vspace{2mm}

$
\Leftrightarrow
\left\{\begin{matrix*}[l]
   2x - y - z + 4 = 0\\
   \\
   x=2t-2\\
   \\
   y= 7t-3 \\
   \\
   z = t
\end{matrix*}\right.
\Leftrightarrow
\left\{\begin{matrix*}[l]
   2(2t-2) - (7t-3) - t + 4 = 0 \\
   \\
   x=2t-2\\
   \\
   y=7t-3\\
   \\
   z=t
\end{matrix*}\right.
\Leftrightarrow
\left\{\begin{matrix*}[l]
   4t - 4 - 7t + 3 - t + 4 = 0 \\
   \\
   x=2t-2\\
   \\
   y=7t-3\\
   \\
   z=t
\end{matrix*}\right.
$

\vspace{2mm}

$
\Leftrightarrow
\left\{\begin{matrix*}[l]
   4t - 4 - 7t + 3 - t + 4 = 0 \\
   \\
   x=2t-2\\
   \\
   y=7t-3\\
   \\
   z=t
\end{matrix*}\right.
\Leftrightarrow
\left\{\begin{matrix*}[l]
   -4t + 3 = 0 \\
   \\
   x=2t-2\\
   \\
   y=7t-3\\
   \\
   z=t
\end{matrix*}\right.
\Leftrightarrow
\left\{\begin{matrix*}[l]
   t = \displaystyle\frac{3}{4} \\
   \\
   x=2\times\displaystyle\frac{3}{4} - 2\\
   \\
   y=7\times\displaystyle\frac{3}{4} - 3\\
   \\
   z= \displaystyle\frac{3}{4}
\end{matrix*}\right.
\Leftrightarrow
\left\{\begin{matrix*}[l]
   x=-\displaystyle\frac{1}{2}\\
   \\
   y=\displaystyle\frac{9}{4}\\
   \\
   z=\displaystyle\frac{3}{4}
\end{matrix*}\right.
$

\vspace{2mm}

Donc la droite $\mathscr{D}$ coupe le plan $(ABC)$ au point d'intersection $I$ de coordonées $I(-\frac{1}{2}, \frac{9}{4}, \frac{3}{4})$

\end{document}

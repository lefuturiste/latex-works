\documentclass{article}
\usepackage[english]{babel}
\usepackage[utf8]{inputenc}
\usepackage{fancyhdr}
\usepackage{amsmath}
\usepackage{amsfonts}
\usepackage{mathrsfs}
\usepackage{mathtools}
\usepackage{indentfirst}
\usepackage{hyperref}
\usepackage{tikz,tkz-tab,amsmath}
\usetikzlibrary{trees}
\usepackage{minted}

\hypersetup{
    colorlinks=true,
    linkcolor=blue,
    filecolor=magenta,      
    urlcolor=blue,
}
\urlstyle{same}

\parskip 1ex

\usepackage{geometry}
 \geometry{
 a4paper,
 left=20mm,
 top=20mm,
 bottom=20mm,
 right=20mm
 }

\pagestyle{fancy}
\fancyhf{}
\rhead{DM 19}
\lhead{Ex 23 p.169}
\rfoot{Page \thepage}

\makeatletter
\def\@seccntformat#1{%
  \expandafter\ifx\csname c@#1\endcsname\c@section\else
  \csname the#1\endcsname\quad
  \fi}
\makeatother

\newcommand{\vspacem}{\vspace{2mm}}
\newcommand{\bfrac}[2]{\displaystyle\frac{#1}{#2}}

\begin{document}

\textbf{1.a}
\vspacem

\noindent La fonction $G$ est est dérivable sur $\mathbb{R}$ 

\noindent  $G(x)$ est de forme $\lambda u(x)$, avec $u(x) = e^{(x)^2}$, $\lambda = \displaystyle\frac{1}{2}$

\noindent $u(x)$ est de forme $e^{v(x)}$, avec $v(x) = x^2$.

$v'(x) = 2x$

\noindent Donc $u'(x) = v'(x)\times e^{v(x)} = 2x\times e^{x^2}$

\noindent Donc $G'(x) = \lambda \times u'(x) = \bfrac{1}{2} \times 2x \times e^{x^2} = x e^{x^2}$

\noindent Donc la fonction $G$ définie sur $\mathbb{R}$ par $G(x)$ est une primitive sur $\mathbb{R}$ de la fonction $g$.

\vspacem
\textbf{1.b}
\vspacem

\noindent $I_1 = \displaystyle\int^1_0 x^1 e^{x^2} \space dx = \displaystyle\int^1_0 xe^{x^2} dx$

\noindent $I_1 = \Big[ G(x) \Big]^1_0 = G(1) - G(0) = \bfrac{1}{2}e^{1^2} - \bfrac{1}{2}e^{0^2} = \bfrac{1}{2} e - \bfrac{1}{2} = \bfrac{1}{2}(e-1)$

\vspacem
\textbf{1.c}
\vspacem

\noindent $H_n$ est dérivable sur $\mathbb{R}$.

\noindent $H_n$ est de forme $u(x) \times v(x)$. Avec :

$u(x) = x^{n+1}$

$u'(x) = (n+1) x^n$

$v(x) = G(x) = \bfrac{1}{2} e^{x^2}$

$v'(x) = g(x) = xe^{x^2}$

\noindent Donc :

$H_n'(x) = u'(x) v(x) + v'(x) u(x)$

$\Leftrightarrow H_n'(x) = \bfrac{n+1}{2}  x^n e^{x^2} + x^{n+2} e^{x^2}$

$\Leftrightarrow \displaystyle\int_0^1 H_n'(x) \space dx = \displaystyle\int_0^1 \frac{n+1}{2} x^n e^{x^2} +  x^{n+2} e^{x^2} \space dx$

$\Leftrightarrow \displaystyle\int_0^1 H_n'(x) \space dx = \displaystyle\int_0^1 \frac{n+1}{2} x^n e^{x^2} \space dx + \displaystyle\int_0^1  x^{n+2} e^{x^2} \space dx$ 

% (l'intégrale de la somme et la somme des intégrales)

$\Leftrightarrow \displaystyle\int_0^1 H_n'(x) \space dx = \bfrac{n+1}{2} \displaystyle\int_0^1 x^n e^{x^2} \space dx + \displaystyle\int_0^1  x^{n+2} e^{x^2} \space dx$

% (propriété linaire)

$\Leftrightarrow \displaystyle\int_0^1 H_n'(x) \space dx = \bfrac{n+1}{2} I_n + I_{n+2}$

$\Leftrightarrow I_{n+2} = H_n(1) - H_n(0) - \bfrac{n+1}{2} I_n$

$\Leftrightarrow I_{n+2} = \bfrac{1}{2} e^{1^2} 1^{n+1} - \bfrac{1}{2} e^{0^2} x^{n+1} - \bfrac{n+1}{2} I_n$

$\Leftrightarrow I_{n+2} = \bfrac{1}{2} e - \bfrac{1}{2} \times 0 - \bfrac{n+1}{2} I_n$

$\Leftrightarrow I_{n+2} = \bfrac{1}{2} e - \bfrac{n+1}{2} I_n$

\vspacem
\textbf{1.d}
\vspacem

$I_3 = I_{1+2} = \bfrac{1}{2} e - \bfrac{1+1}{2} \times I_1 = \bfrac{1}{2} e -\Big(\bfrac{1}{2} e - \bfrac{1}{2}\Big) = \bfrac{1}{2}e - \bfrac{1}{2}e + \bfrac{1}{2} = \bfrac{1}{2}$

$I_5 = I_{3+2} = \bfrac{1}{2} e - \bfrac{3+1}{2} \times I_3 = \bfrac{1}{2} e - 2 \times \bfrac{1}{2} = \bfrac{1}{2} e - 1$

\vspacem
\textbf{2.}
\vspacem

\noindent Pour trouver la réponse à cette question, il suffit d'exécuter l'algorithme en changeant la sortie pour que le programme affiche la valeur de la variable $n$

\inputminted{python}{main.py}

\noindent C'est le terme $I_{21}$  que l'on obtient à la sortie de cet algorithme.

\vspacem
\textbf{3.a}
\vspacem

\noindent Une intégrale est positive si 1: ses bornes sont dans l'ordre croissant (ce qui est le cas ici) 2: la fonction est positive dans les bornes de l'intégrale. 

\noindent On a $x^n \geq 0$ et $e^{x^2} > 0$, donc $x^n e^{x^2} \geq 0$. D'où $I_n \geq 0$ pour tout entier naturel non nul.

\vspacem
\textbf{3.b}
\vspacem

$x \leq 1 $

$\Leftrightarrow x^{n+1} \leq x^n$

$\Leftrightarrow x^{n+1}e^{x^2} \leq x^n e^{x^2}$

$\Leftrightarrow I_{n+1} \leq I_n$

\noindent L'équation $I_{n+1} \leq I_n$ signifie que la suite $I_n$ est décroissante.

\iffalse
\begin{gather*}
x \leq 1\\
\Leftrightarrow x^{n+1} \leq x^n\\
\Leftrightarrow x^{n+1}e^{x^2} \leq x^n e^{x^2}\\
\Leftrightarrow I_{n+1} \leq I_n
\end{gather*} 
\fi

\vspacem
\textbf{3.c}
\vspacem

\noindent Du fait que la suite $(I_n)$ est minorée par 0 (question 3.a) et qu'elle est décroissante (question 3.b) alors on peut en déduire que la suite est convergente vers une limite $l$.

\vspacem
\textbf{4.}
\vspacem

\noindent Premièrement, on doit trouver l'encadrement de la fonction $x^n e^{x^2}$ qui est intégré dans la suite : 

\noindent Avec $x \in [0, 1]$, on a $x^2 \leq 1$ donc $e^{x^2} \leq e$. On a aussi également $x^n \geq 0$. Donc :

$x^n e^{x^2} \leq ex^n$

\noindent On utilise la propriété de la conservation de l'ordre de l'intégrale. Si dans l'intervalle de 0 à 1 on a $x^n e^{x^2} \leq ex^n$, alors on a aussi :

$\displaystyle\int^1_0 x^n e^{x^2} dx \leq \displaystyle\int^1_0 e x^n dx$

\noindent On cherche l'intégrale $\int^1_0 e x^n dx$ :

\noindent Une primitive de $k u(x)$ (avec $k$ constante) est $kU(x)$. Une primitive de $x^n$ est $\bfrac{x^{n+1}}{n+1}$. Donc une primitive de $e x^n$ est $e \bfrac{x^{n+1}}{n+1}$. Ce qui donne pour l'intégrale :

$\displaystyle\int^1_0 e x^n dx = \Bigg[ e \bfrac{x^{n+1}}{n+1} \Bigg]^1_0 = e\bfrac{1^{n+1}}{n+1} - e\bfrac{0^{n+1}}{n+1} = \bfrac{e}{n+1}$

\noindent Au final, on peut encadrer tout terme de la suite $(I_n)$ pour tout entier naturel non nul:

$0 \leq I_n \leq \bfrac{e}{n+1}$

\noindent Grâce à cet encadrement on peut déterminer la limite avec le théorème des gendarmes :

\[
    \left.\begin{matrix*}[r]
       0 \leq I_n \leq \bfrac{e}{n+1}\\
       \\
       \\
       \displaystyle\lim_{n \to + \infty} 0 = 0\\
       \\
       \\
       \displaystyle\lim_{n \to + \infty} \bfrac{e}{n+1} = 0
    \end{matrix*}\medspace\right\}
    \left.\begin{matrix*}
        \space\text{D'après le théorème des gendarmes :}\vspace{2mm}\\
        \displaystyle\lim_{n\rightarrow+\infty} I_n = 0
    \end{matrix*}\right.
\]

Donc la suite $(I_n)$ est convergente et a pour limite $l = 0$.



\iffalse

$\int^1_0 H_n \space dx = H_n(1) - H_n(0)$
On cherche l'intégrale de $H_n'$
$H_n'(x) = e^{x^2} \Bigg(\bfrac{1}{2} (n+1) x^n + x^2 \times x^{n} \Bigg)$

$H_n'(x) = e^{x^2} x^n \Bigg(\bfrac{n+1}{2} + x^2 \Bigg)$


$I_{n+2} = \displaystyle\int^1_0 x^{n+2} e^{x^2} \space dx$

$I_{n+2} = \displaystyle\int^1_0 x^{n+1} x e^{x^2} \space dx$

$I_{n+2} = \displaystyle\int^1_0 x^{n+1} g(x) \space dx$

\noindent La fonction à intégrer est de forme $u$

\iffalse
$h(x) = x^{n+2} e^{x^2}$
\fi

\newpage
\vspace{5mm}

\noindent On considère la suite $(I_n)$ définie pour $n$ entier naturel non nul par :

$I_n = \displaystyle\int^1_0 x^n e^{x^2} \space dx$

\noindent Soit la fonction $g$ définie sur $\mathbb{R}$ par $g(x) = xe^{x^2}$

\noindent Soit la fonction $G$,  une primitive sur $\mathbb{R}$ de la fonction $g$, définie sur $\mathbb{R}$ par $G(x) = \bfrac{1}{2} e^{x^2}$

\noindent Pour tout entier naturel $n$, on définit sur $\mathbb{R}$ la fonction $H_n$ par :

\vspace{1mm}
$H_n(x) = x^{n+1} G(x)$
\vspace{1mm}

\noindent Montrer que $H_n$ est dérivable sur $\mathbb{R}$, calculer pour tout réel $x$, $H_n'(x)$ et en déduire que, pour tout entier naturel $n$ :

\vspace{1mm}
$I_{n+2} = \bfrac{1}{2} e - \bfrac{n+1}{2} I_n$

\fi


\end{document}
